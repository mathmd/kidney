% !TEX program = xelatex
\documentclass{article}

%% ---- Set up paper margin ---- %%
%\usepackage[a4paper,top=1in,bottom=1in,left=1in,right=1in]{geometry}

% use multiple languages
%% ---- allow CJK usage ---- %%
\usepackage[CJKspace]{xeCJK} % this should be called before Polyglossia
\setCJKmainfont{Noto Serif CJK TC}
\setCJKsansfont{Noto Sans CJK TC}
\setCJKmonofont{Noto Sans Mono CJK TC}
% \setCJKmainfont{Noto Serif JP}
% \setCJKsansfont{Noto Sans JP}
% \setCJKmonofont{Noto Sans Mono CJK JP}

%% ---- ตั้งค่าให้ตัดคำภาษาไทย ---- %%
\XeTeXlinebreaklocale "th"
\XeTeXlinebreakskip = 0pt plus 0pt % เพิ่มความกว้างเว้นวรรคให้ความยาวแต่ละบรรทัดเท่ากัน

%% ---- font settings ---- %%
\usepackage{fontspec}
\defaultfontfeatures{Mapping=tex-text} % map LaTeX formating, e.g., ``'', to match the current font
% To change the main font, uncomment one of the below command.
% \setmainfont{TeX Gyre Termes} % Free Times
% \setsansfont{TeX Gyre Heros} % Free Helvetica
% \setmonofont{TeX Gyre Cursor} % Free Courier
\newfontfamily{\thaifont}[Scale=MatchUppercase,Mapping=textext]{Laksaman} % ตั้งฟอนต์หลักภาษาไทย
\newenvironment{thailang}{\thaifont}{} % create environment for Thai language
\usepackage[Latin,Thai]{ucharclasses} % ตั้งค่าให้ใช้ "thailang" environment เฉพาะ string ที่เป็น Unicode ภาษาไทย
\setTransitionTo{Thai}{\begin{thailang}}
\setTransitionFrom{Thai}{\end{thailang}}

%% ---- spacing between lines ---- %%
\usepackage{setspace}
% \singlespacing % default setting
% \onehalfspacing % recommend using this for Thai language

%% ---- using alphabatic language ---- %%
\usepackage{polyglossia}
\setdefaultlanguage{english} % it is preferrable to set English as the main language, since the numeric system is compatible with most LaTeX features such as 'enumerate' and so on
\setotherlanguages{thai}

\AtBeginDocument\captionsthai % allow captions to be in Thai


%% ---- math packages ---- %%
\usepackage{amsmath}
\usepackage{amssymb}
\usepackage{bm} % same functionality as \mathbf{} but for greek letters
\usepackage{diffcoeff}
\usepackage{physics}
\usepackage{mathdots}
% \numberwithin{equation}{section} % equation numbers are formatted as <#Section>.<#eq in the section>

%% ---- define math environment ---- %%
\usepackage{amsthm}
% \newtheorem{definition}{Definition}[section]
\newtheorem{definition}{Definition}
\newtheorem{proposition}[definition]{Proposition}
\newtheorem{theorem}[definition]{Theorem}
\newtheorem{corollary}{Corollary}[definition]
\newtheorem{remark}{Remark}[definition]
\newtheorem{lemma}[definition]{Lemma}
\newtheorem{problem}[definition]{Problem}
% \newtheorem*{problem}{Problem}

%% ---- hyperref settings ---- %%
\usepackage{hyperref}
\usepackage{url}
\usepackage{cite}
\usepackage{xcolor}
\hypersetup{
    colorlinks,
    linkcolor={red!50!black},
    citecolor={green!50!black},
    urlcolor={blue!80!black}
    }

%% ---- misc. packages ---- %%
\usepackage{enumitem} % allow customizing list environments: enumerate, itemize and description.
\usepackage{mhchem} % use chemistry notation
\usepackage{lipsum}
\usepackage{metalogo} % for extended LaTeX logo such as XeTeX
\usepackage{subcaption} % allowing subfigure environment
% \usepackage[section]{placeins} % ensure floats do not go into the next section and allow the use of \FloatBarrier
\usepackage{graphicx} % allow cropping and rotating images
\usepackage{caption}
\usepackage{float} % allow the use of [H] for positioning of tables and figures
\usepackage{tikz-cd}
\restylefloat{table}

%% ---- title, authors, and dates ---- %%
\usepackage{authblk}
\title{Multidomain model of kidney}
\author[1]{Chanoknun Sintavanuruk}
% \affil[1]{Department of Physiology, Faculty of Medicine Siriraj Hospital, Mahidol University}
% \author[2,3]{XXXX XXXX}
% \affil[2]{Department of XXXX, XXXX University}
% \affil[3]{Department of XXXX, XXXX University}
\date{\today}

\begin{document}
\sloppy % ช่วยตัดคำภาษาไทย
\maketitle

This is a reformulation of Alan Weinstein's kidney model as of 2022 in the framework of multidomain model.
In the process, we point out the necessary assumptions, in addition to those of multidomain model, needed to be made in order to derive the kidney model.
Then, we give the steady state model used in Alan Weinstein's recent papers.
This is in hope that we have a clear overall picture which will ease us in the analysis of the model.

\section{Model derivation}

Consider a one-dimensional domain $\Omega = (-2,7)$ which represents the radial coordinate of the kidney tissue from the superficial to the deeper layers; here, we assume that there is no variation across the kidney tissue of the same depth.
Specifically, $(-2,0)$ represents medullary ray, and $[0,7)$ represents renal medulla.
The domain is multiphasic in the sense that multiple radially aligned compartments of the kidney --- which include the lumina, tubular cells and their intercellular space (also called lateral interspace) of different tubular segments (except convoluted tubules and connecting tubules), interstitia of medulla together with medullary ray, and vasa recta --- are formally described within the same region, $\Omega$.

Anatomically speaking, the renal cortex also share the same region $(-2,0)$ as medullary ray.
However, since the geometry of the convoluted tubules and connecting tubules, which are contained in the renal cortex, are difficult to describe, the spatial variations within the interstitium and the capillary plexus of the renal cortex are ignored.
Therefore, we treat the tubules within the cortex as separated systems with their own one-dimensional domain which are coupled to components in $\Omega$ via their boundary and the cortical interstitium; we will come back to this later.

We will use the label
% $k=\pm(3N+1)$ the ascending and the descending vasa recta respectively and 
$k=0$ for the interstitium of medullary ray and the renal medulla.
It is important to note that, in this model, we do not make a distinction between the actual interstitium and the capillary plexus present in the renal cortex and medullary ray.
Suppose there are $N$ distinct types of nephrons.
For each type of nephron there are $2$ components contained in $\Omega$: the descending and the ascending parts each of which are also divided into the lumen, the intracellular space (ICS), and the lateral intercellular space (LIS).
For the $n^{\mathrm{th}}$ type nephron, $1\leq n\leq N$, we label by $k=\pm n$, $k=\pm(N+n)$, and $k=\pm(2N+n)$ the descending (positive) and the ascending (negative) parts of the lumen, the ICS, and the LIS respectively.
Similarly, for $P$ distinct types of vasa recta, we label by $k=\pm(3N+j)$, for $j=1,\dots,P$, the descending and the ascending vasa recta.
Finally, the lumen, the ICS, and the LIS of the collecting tubules are labeled by $k=3N+P+1,3N+P+2,3N+P+3$ respectively.

We describe the occupied volume of each $k^{\mathrm{th}}$ compartment, $-3N-P\leq k\leq 3N+P+3$, in $\Omega$ by the volume per unit depth $\alpha_k:\Omega_k\times [0,\tau)\to \mathbb{R}_+$, where $0<\tau\leq \infty$ and $\Omega_k\subset\Omega$ is an open interval with $\Omega_{n} \supset \Omega_{N'+n} = \Omega_{2N'+n}$ and $\inf\Omega_n = \inf\Omega_{N'+n}\leq 0$ for $n=\pm 1,\dots\pm N$ where we denote $N'+n := \mathrm{sign}(n)N+n$ and $2N'+n := 2\, \mathrm{sign}(n)N+n$.
The sets $\Omega_n\setminus \Omega_{N'+n}$ are where the thin loops of Henle located.
Additionally, each loop of vasa recta begins at and returns to the cortex-medullary junction at $x=0$, i.e., $\inf \Omega_{3N+j} = \inf \Omega_{-3N-j}=0$, for $j=1,\dots,P$.
Moreover, we must have the turning points of the loops of Henle and vasa recta at the end and the beginning of their corresponding descending and the ascending compartments, i.e., $\sup\Omega_k = \sup\Omega_{-k}$ for $k=1,\dots,N,3N+1,\dots,3N+P$.
For collecting tubules, we have $\Omega_{3N+P+1}=\Omega_{3N+P+2}=\Omega_{3N+P+3}=\Omega$.
We require that, by setting $\alpha_k=0$ in $\Omega\setminus\Omega_k$, we must have
\begin{equation}\label{eq:const_vol}
    \sum_{k=-3N-P}^{3N+P+3} \alpha_k = \alpha
\end{equation}
    for a given volume density function $\alpha:\overline{\Omega}\to\mathbb{R}_+$.
That is, we assume that the total volume of the medullary tissue is constant.
Note that, in the steady state model of kidney, $\alpha_k$ is only dependent on the depth $x\in \Omega_k$.
% Therefore, we can effectively treat $\alpha_k$ as the dimensionless volume of the compartment $k$.

For time-dependent model and $n=\pm 1,\dots,\pm N$, we have the equations for the tubular compartments:
\begin{align}
    \frac{\partial \alpha_n}{\partial t} + \frac{\partial}{\partial x}(\alpha_nu_n) &= -\gamma_n(w_{1}^n+w_{2}^n)\quad \quad \quad \ \ \text{in}\quad \Omega_{n},\\
    \frac{\partial \alpha_{N'+n}}{\partial t} &= \gamma_{n}w_{1}^n-\beta_nv_1^n-\lambda_n\ell_0^n\quad \text{in}\quad \Omega_{N'+n},\\
    \frac{\partial \alpha_{2N'+n}}{\partial t} &= \gamma_{n}w_{2}^n-\beta_nv_2^n+\lambda_n\ell_0^n\quad \text{in}\quad \Omega_{2N'+n}=\Omega_{N'+n},\\
    \gamma_n(w_m^n) &= \beta_n(v_m^n),\quad m=1,2,\quad \ \text{in}\quad \Omega_n\setminus\Omega_{N'+n}.
    % \alpha_n = \alpha_{-n},&\quad u_n = -u_{-n}\qquad\qquad\quad \ \text{on}\ \ \sup\Omega_n
\end{align}
% and the ascending tubular compartments:
% \begin{align}
%     \frac{\partial \alpha_{-n}}{\partial t} + \frac{\partial}{\partial x}(\alpha_{-n}u_{-n}) &= -\gamma_{-n}(w_{1}^{-n}+w_{2}^{-n})\quad \quad \quad \ \ \text{in}\quad \Omega_{-n},\\
%     \frac{\partial \alpha_{-N-n}}{\partial t} &= \gamma_{-n}w_{1}^{-n}-\ell_0^{-n}-\beta_{-n}v_1^{-n}\quad \text{in}\quad \Omega_{-N-n},\\
%     \frac{\partial \alpha_{-2N-n}}{\partial t} &= \gamma_{-n}w_{2}^{-n}+\ell_0^{-n}-\beta_{-n}v_2^{-n}\quad \text{in}\quad \Omega_{-2N-n},
% \end{align}
Here, $u_n$ is the flow velocity of the water in the lumen;
$\gamma_n,\beta_n,\lambda_n:\overline{\Omega_n}\to \mathbb{R}_+$ represent the area per unit depth of the luminal membrane, the basement membrane, and the cell membrane between the ICS and the LIS;
$w^n_m,v^n_m$ are the water flux (per unit depth) from the lumen and into the interstitium respectively with the subscript $m=1,2$ represents into or from the ICS and the LIS respectively, and $\ell_0^n$ is the water flux from the ICS into LIS.
We will give the equations for $u_n,v_n,w_n,\ell_0^n$ later.

Similarly, for the collecting tubules ($k=K,K+1,K+2$, $K:=N+P+1$):
\begin{align}
    \frac{\partial \alpha_K}{\partial t} + \frac{\partial}{\partial x}(\alpha_Ku_K) &= -\gamma_K(w_{1}^K+w_{2}^K)\qquad \qquad \, \text{in}\quad \Omega,\\
    \frac{\partial \alpha_{K+1}}{\partial t} &= \gamma_Kw_{1}^K-\beta_Kv_1^K-\lambda_K\ell_0^K\quad \text{in}\quad \Omega,\\
    \frac{\partial \alpha_{K+2}}{\partial t} &= \gamma_Kw_{2}^K-\beta_Kv_2^K+\lambda_K\ell_0^K\quad \text{in}\quad \Omega,
\end{align}
where $u_K,\gamma_K,\beta_K,\lambda_K,w_1^K,w_2^K,v_1^K,v_2^K,\ell^K_0$ are interpreted the same as previously.

For vasa recta ($k=3N'+j:=3\,\mathrm{sign}(j)N+j$, $j=\pm 1,\dots,\pm P$) and the interstitium ($k=0$), we have:
\begin{align}
    \frac{\partial \alpha_{3N'+j}}{\partial t} + \frac{\partial}{\partial x}(\alpha_{3N'+j}u_{3N'+j}) = -\eta_{j}\omega_{j}\qquad &\text{in}\quad \Omega_{3N'+j},\\
    \frac{\partial \alpha_0}{\partial t} + \frac{\partial}{\partial x}(\alpha_0u_0) = 
        \sum_{\substack{1\leq|n|\leq N\\\text{or } n=K}}\beta_n(v_{1}^n+v_{2}^n)+\sum_{|j|\leq P+1}\eta_j\omega_j \quad &\text{in}\quad \Omega.\label{eq:s_water}
    % \alpha_n = \alpha_{-n},\quad u_n = -u_{-n}\qquad\qquad\quad \ &\text{on}\ \ \sup\Omega_n
\end{align}
Here, $u_0,u_{3N'+j}$ are also the flow velocity, $\omega_j$ is the water flux from the vessel ($j=\pm 1,\dots,\pm P$), the cortex ($j=0$) into the interstitium, and the input and output ($j=\pm(P+1)$) into or from the capillary plexus of medullary ray.
For convenience, we set $\omega_j = 0$ in $\Omega\setminus \Omega_{3N'+j}$, $j=\pm 1,\dots,\pm P$, and $v_1^n,v_2^n=0$ in $\Omega\setminus\Omega_n$.
In Alan Weinstein's steady state model, $\eta_{P+1}\omega_{P+1}$ is given and $\eta_{-P-1}\omega_{-P-1}$ is computed so that the balance equation is satisfied, but more generally, we can have them depending on pressures which we will describe later.
The parameter $\eta_j:\overline{\Omega_{3N'+j}}\to\mathbb{R}_+$, $j=\pm 1,\dots,\pm P$, are the area per unit depth of the wall of vasa recta.
%  and $\eta_0\in C(\overline{\Omega})$ is that of the cortex-medullary ray junction.
Note also that, in Alan Weinstein's steady state model 
% (obtained by omitting all the $\partial/\partial t$ terms), 
the left-hand side of (\ref{eq:s_water}) is zero by an assumption that $u_0$ is identically zero in $\Omega$.
Additionally, it is also assumed that $\omega_0\equiv 0$ in his model, i.e., the interstitia of the cortex and medullary ray are completely separated.

% Further, we have the condition at the turning point of loops of Henle and vasa recta:
% \begin{equation}
%     \alpha_ku_k = -\alpha_{-k}u_{-k},\quad\text{on}\quad\sup\Omega_k,\quad k= 1,\dots,3N+P.
% \end{equation}

Now, let there be $M$ mobile solute species, e.g., \ce{Na+}, \ce{K+}, \ce{Cl-}, glucose, \ce{CO2}, etc., which are labeled by $i=1,\dots,M$.
We denote by $c_i^k:\Omega_k\times [0,T)\to \mathbb{R}_+$ the concentration of the $i^{\mathrm{th}}$ solute in the $k^{th}$ compartment, and by $a_i^k:=\alpha_kc_i^k$ the solute amount of $i$ in $k$ per unit depth.
We have the equations for the solutes in nephron compartments, with $n=\pm 1,\dots,\pm N$ and $i=1,\dots,M$:
\begin{align}
    \frac{\partial }{\partial t}(\alpha_nc_i^n) = -\frac{\partial f_i^n}{\partial x}-\gamma_n(g_i^{n,1}+g_i^{n,2})+s_i^{n} \qquad \ &\text{in}\quad \Omega_{n},\\
    \frac{\partial }{\partial t}(\alpha_{N'+n}c_i^{N'+n}) = \gamma_{n}g_i^{n,1}-\beta_nq_i^{n,1}-\lambda_n\ell_i^n+s_i^{N'+n} \quad \ &\text{in}\quad \Omega_{N'+n},\\
    \frac{\partial }{\partial t}(\alpha_{2N'+n} c_i^{2N'+n}) = \gamma_{n}g_i^{n,2}-\beta_nq_i^{n,2}+\lambda_n\ell_i^n+s_i^{2N'+n}\quad &\text{in}\quad \Omega_{N'+n},\\
    \gamma_ng_i^{n,m} = \beta_nq_i^{n,m},\quad m=1,2, \qquad \qquad \quad \text{in}&\quad \Omega_n\setminus\Omega_{N'+n},
\end{align}
    where $g_i^{n,m},q_i^{n,m}$ are the solute flux (per unit depth) from the lumen and into the interstitium with the superscript $m=1,2$ represents into or from the ICS and the LIS respectively, $\ell_i^n$ is the solute flux from the ICS into LIS, and $s_i^{k}$ is the generation of the solute $i$ in the $k^{\mathrm{th}}$ compartment.
Similarly, for collecting tubules, we also have
\begin{align}
    \frac{\partial }{\partial t}(\alpha_Kc_i^K) = -\frac{\partial f_i^K}{\partial x}-\gamma_K(g_i^{n,1}+g_i^{n,2})+s_i^K \qquad \ &\text{in}\quad \Omega,\\
    \frac{\partial }{\partial t}(\alpha_{K+1}c_i^{K+1}) = \gamma_Kg_i^{n,1}-\beta_Kq_i^{n,1}-\lambda_K\ell_i^K+s_i^{K+1} \quad \ &\text{in}\quad \Omega,\\
    \frac{\partial }{\partial t}(\alpha_{K+2} c_i^{K+2}) = \gamma_Kg_i^{n,2}-\beta_Kq_i^{n,2}+\lambda_K\ell_i^K+s_i^{K+2}\quad &\text{in}\quad \Omega.
\end{align}

We will use $i=M$ to represent \ce{CO2} and write $\mathbf{s}_k = (s_1^k,\dots,s_M^k)^\top$ for all tubular compartments $k=\pm 1,\dots,\pm 3N,\, 3N+P+1,\, 3N+P+2,\, 3N+P+3$.
We describe the tubular solute generation by an invertible reaction matrix $R\in \mathbb{Z}^{M\times M}$ and the metabolic generation $\mathbf{r}_k$, whose value is in $\mathbb{R}^M$ so that we have
\begin{equation}\label{eq:gen}
    R \mathbf{s}_k = \mathbf{r}_k.
\end{equation}
The trivial case of $R$ would be $R=I$, the identity matrix, and $\mathbf{r}_k=\mathbf{0}$.
In case that there are buffer reactions, $R$ represents both mass conservation and the equilibria of reactions whose kinetics are assumed to be instantaneous; otherwise, the reaction kinetics will be captured by $\mathbf{r}_k$.
In general, $\mathbf{r}_k$ is a function depending on $\mathbf{c}_k:=(c_1^k,\dots,c_M^k)^\top$, and the fluxes of active membrane transports when $k$ is a cellular compartment which generates \ce{CO2} via oxidative metabolism.

For vasa recta ($k=3N'+j,\,j=\pm 1,\dots,\pm P$) and the interstitium ($k=0$) --- recall that we treat the interstitium and the capillary plexus of medullary ray as a single compartment, there are additional solute species, namely impermeable proteins and Hemoglobin buffer species which we label by $i=M+1,\dots,M+B$.
We have the solute equations, with $i=1,\dots,M+B$, for vasa recta:
\begin{align}
    \frac{\partial }{\partial t}(\alpha_{3N'+j}c_i^{3N'+j}) = - \frac{\partial f_i^{3N'+j}}{\partial x} -\eta_{j}\vartheta_i^j+s_i^{3N'+j}\qquad &\text{in}\quad \Omega_{3N'+j}.
    % \\
    % \frac{\partial}{\partial t}(\alpha_0 c_i^0) = - \frac{\partial f_i^0}{\partial x} + 
    %     \sum_{\substack{1\leq|n|\leq N\\\text{or } n=K}}\beta_n(q_{1}^n+q_{2}^n)+\sum_{0\leq |j|\leq P}\eta_j\vartheta_i^j\quad &\text{in}\quad \Omega.
    % \alpha_n = \alpha_{-n},\quad u_n = -u_{-n}\qquad\qquad\quad \ &\text{on}\ \ \sup\Omega_n
\end{align}
For the interstitium, we have the equations for $i=1,\dots,M$:
\begin{equation}
    \frac{\partial}{\partial t}(\alpha_0 c_i^0) = - \frac{\partial f_i^0}{\partial x} + 
        \sum_{\substack{1\leq|n|\leq N\\\text{or } n=K}}\beta_n(q_{1}^n+q_{2}^n)+\sum_{|j|\leq P+1}\eta_j\vartheta_i^j+s_i^0\quad \text{in}\quad \Omega,
\end{equation}
    and for $i=M+1,\dots,M+B$:
\begin{equation}
    \frac{\partial}{\partial t}(\alpha_0 c_i^0) = - \frac{\partial f_i^0}{\partial x} + 
    \sum_{j=\pm(P+1)}\eta_j\vartheta_i^j+s_i^0\quad \text{in}\quad (-2,0).
\end{equation}
Here, $\vartheta_i^j$ is the solute flux from vasa recta ($j=\pm 1,\dots,\pm P$) and the renal cortex ($j=0$) into the interstitium, and the input and output ($j=\pm(P+1)$) into or from the capillary plexus of medullary ray.
We also set $\vartheta_i^j=0$ in $\Omega\setminus \Omega_{3N'+j}$ for $j=\pm 1,\dots,\pm P$, and $q_1^n,q_2^n=0$ in $\Omega\setminus\Omega_n$.
Note that it is assumed $f_i^0\equiv 0$ and $\vartheta_i^j\equiv 0$ in Alan Weinstein's model.
Moreover, since there is no need to explicitly describe $\vartheta_i^j$, $j=\pm(P+1)$ in the steady state model, $\vartheta_i^{P+1}$ is given and $\vartheta_i^{-P-1}$ is a model unknown in such a case; we need to have equations for these for the time-dependent model.

The reaction terms $s_i^k$, $k=0,\pm(3N+1),\dots,\pm(3N+P)$, also have similar equations as (\ref{eq:gen}), but now with $\mathbf{s}_k:=(s_1^k,\dots,s_M^k,s_{M+1}^k,\dots,s_{M+B}^k)$, $R$ is replaced by another invertible matrix $R'\in \mathbb{R}^{(M+B)\times (M+B)}$ and $\mathbf{r}_k$ has value in $\mathbb{R}^{M+B}$, i.e.,
\begin{equation}
    R'\mathbf{s}_k=\mathbf{r}_k,\quad k=0,\pm(3N+1),\dots,\pm(3N+P).
\end{equation}

Now, we describe the water flow velocity, $u_k$ and the solute flow within each compartment $f_i^k$.
In Alan Weinstein's steady state model, there is no axial flow in the interstitium ($k=0$) --- that is $f_i^0=u_0=0$ for all $i$.
For $k=\pm 1,\dots,\pm N,\pm(3N+1),\dots,\pm(3N+P),K,K+1,K+2$, 
the water flow is given by Poisseuille's equation $\rho_k\alpha_ku_k = -\partial p_k/\partial x$, where $\rho_k\alpha_k$ is the hydrolic resistivity with $\rho_k$ constant and $p_k$ is the hydrostatic pressure, and the corresponding solute flows 
$f_i^k$ are assumed to be purely convective, i.e., $f_i^k = \alpha_ku_kc_i^k$.
In general, $f_i^k$ can be generalized into Nernst-Planck equation:
\begin{equation}
    f_i^k = -D_i^k\left( \frac{\partial c_i^k}{\partial x}+\frac{z_iFc_i^k}{RT} \frac{\partial\phi_k}{\partial x}\right) + \alpha_k u_kc_i^k,
\end{equation}
where $D_i^k$ is the diffusion coefficient, $z_i$ is the valence of the solute $i$, $F/RT$ is a constant, $\phi_k$ is the electrical potential in compartment $k$;
% Nevertheless, we need additional equations to describe the water flow in order to have a complete time-dependent model.
and the water flow velocity $u_k$
% , including possibly when $k=0$, 
is determined by the compartmental pressure field and driven by electrostatic force:
\begin{gather}
    \rho_k\alpha_k u_k = -\frac{\partial \tilde{p}_k}{\partial x} - \sum_{i=1}^{M'}z_iFc_i^k\frac{\partial \phi_k}{\partial x},\\
    M'=\begin{cases}
        M&\quad \text{if}\quad k=\pm 1,\dots,\pm N,K,K+1,K+2,\\
        M+B &\quad \text{if}\quad k=0,\pm(3N+1),\dots,\pm(3N+P),
    \end{cases}\nonumber
\end{gather}
where $\tilde{ p }_k:= p_k-RTa_k/\alpha_k$ is the compartmental pressure with $a_k$ being the amount per unit depth of immobile solute.
Finally, the solute and water flows at the turning points of loops of Henle and vasa recta must match, i.e.,
\begin{align}
    \alpha_ku_k = -\alpha_{-k}u_{-k},\quad\text{on}\quad\sup\Omega_k,\quad k= 1,\dots N,3N+1,\dots,3N+P,\\
    f_i^k = -f_i^{-k},\quad\text{on}\quad\sup\Omega_k,\quad k= 1,\dots N,3N+1,\dots,3N+P.
\end{align}

To determine the electrical potential $\phi_k$ and the hydrostatic pressure in each compartment, we have an electroneutrality approximation:
\begin{equation}
     0=z_0^k Fa_k+\sum_{i=1}^{M'}z_iF \alpha_k c_i^k,
\end{equation}
and the pressure balance between compartments is described by compliances $\nu_k$, $k=\pm 1,\dots,\pm N,\pm(3N+1),\dots,\pm(3N+P),K,K+1,K+2$:
\begin{equation}
    \nu_k(p_k - p_0) = \frac{\alpha_k}{\alpha_k^0}-1,\quad 
\end{equation}
where $\alpha_k^0$ is the baseline volume in which the pressure on both sides are equal.
For the ICS and ECS compartments, we set, for $n=\pm 1,\dots,\pm N$:
\begin{gather}
    p_{N'+n} = p_n,\\
    p_K = p_{K+1},\\
    p_{2N'+n} = p_0 = p_{K+2}.
\end{gather}
Note that, in Alan Weinstein's model, the hydrostatic pressures of vasa recta and interstitium are computed from the balance equation, since the volume densities of vasa recta and interstitium are fixed.
These conditions are functionally equivalent to our equation (\ref{eq:const_vol}).

\end{document}