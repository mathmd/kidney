% !TEX program = xelatex
\documentclass{article}

%% ---- Set up paper margin ---- %%
%\usepackage[a4paper,top=1in,bottom=1in,left=1in,right=1in]{geometry}

% use multiple languages
%% ---- allow CJK usage ---- %%
\usepackage[CJKspace]{xeCJK} % this should be called before Polyglossia
\setCJKmainfont{Noto Serif CJK TC}
\setCJKsansfont{Noto Sans CJK TC}
\setCJKmonofont{Noto Sans Mono CJK TC}
% \setCJKmainfont{Noto Serif JP}
% \setCJKsansfont{Noto Sans JP}
% \setCJKmonofont{Noto Sans Mono CJK JP}

%% ---- ตั้งค่าให้ตัดคำภาษาไทย ---- %%
\XeTeXlinebreaklocale "th"
\XeTeXlinebreakskip = 0pt plus 0pt % เพิ่มความกว้างเว้นวรรคให้ความยาวแต่ละบรรทัดเท่ากัน

%% ---- font settings ---- %%
\usepackage{fontspec}
\defaultfontfeatures{Mapping=tex-text} % map LaTeX formating, e.g., ``'', to match the current font
% To change the main font, uncomment one of the below command.
% \setmainfont{TeX Gyre Termes} % Free Times
% \setsansfont{TeX Gyre Heros} % Free Helvetica
% \setmonofont{TeX Gyre Cursor} % Free Courier
\newfontfamily{\thaifont}[Scale=MatchUppercase,Mapping=textext]{Laksaman} % ตั้งฟอนต์หลักภาษาไทย
\newenvironment{thailang}{\thaifont}{} % create environment for Thai language
\usepackage[Latin,Thai]{ucharclasses} % ตั้งค่าให้ใช้ "thailang" environment เฉพาะ string ที่เป็น Unicode ภาษาไทย
\setTransitionTo{Thai}{\begin{thailang}}
\setTransitionFrom{Thai}{\end{thailang}}

%% ---- spacing between lines ---- %%
\usepackage{setspace}
% \singlespacing % default setting
% \onehalfspacing % recommend using this for Thai language

%% ---- using alphabatic language ---- %%
\usepackage{polyglossia}
\setdefaultlanguage{english} % it is preferrable to set English as the main language, since the numeric system is compatible with most LaTeX features such as 'enumerate' and so on
\setotherlanguages{thai}

\AtBeginDocument\captionsthai % allow captions to be in Thai


%% ---- math packages ---- %%
\usepackage{amsmath}
\usepackage{amssymb}
\usepackage{bm} % same functionality as \mathbf{} but for greek letters
\usepackage{diffcoeff}
\usepackage{physics}
\usepackage{mathdots}
% \numberwithin{equation}{section} % equation numbers are formatted as <#Section>.<#eq in the section>

%% ---- define math environment ---- %%
\usepackage{amsthm}
% \newtheorem{definition}{Definition}[section]
\newtheorem{definition}{Definition}
\newtheorem{proposition}[definition]{Proposition}
\newtheorem{theorem}[definition]{Theorem}
\newtheorem{corollary}{Corollary}[definition]
\newtheorem{remark}{Remark}[definition]
\newtheorem{lemma}[definition]{Lemma}
\newtheorem{problem}[definition]{Problem}
% \newtheorem*{problem}{Problem}

%% ---- hyperref settings ---- %%
\usepackage{hyperref}
\usepackage{url}
\usepackage{cite}
\usepackage{xcolor}
\hypersetup{
    colorlinks,
    linkcolor={red!50!black},
    citecolor={green!50!black},
    urlcolor={blue!80!black}
    }

%% ---- misc. packages ---- %%
\usepackage{enumitem} % allow customizing list environments: enumerate, itemize and description.
\usepackage{mhchem} % use chemistry notation
\usepackage{lipsum}
\usepackage{metalogo} % for extended LaTeX logo such as XeTeX
\usepackage{subcaption} % allowing subfigure environment
% \usepackage[section]{placeins} % ensure floats do not go into the next section and allow the use of \FloatBarrier
\usepackage{graphicx} % allow cropping and rotating images
\usepackage{caption}
\usepackage{float} % allow the use of [H] for positioning of tables and figures
\usepackage{tikz-cd}
\restylefloat{table}

%% ---- title, authors, and dates ---- %%
\usepackage{authblk}
\title{Multidomain model of kidney}
\author[1]{Chanoknun Sintavanuruk}
% \affil[1]{Department of Physiology, Faculty of Medicine Siriraj Hospital, Mahidol University}
% \author[2,3]{XXXX XXXX}
% \affil[2]{Department of XXXX, XXXX University}
% \affil[3]{Department of XXXX, XXXX University}
\date{\today}

\begin{document}
\sloppy % ช่วยตัดคำภาษาไทย
\maketitle

This is a reformulation of Alan Weinstein's kidney model as of 2022 in the framework of multidomain model.
In the process, we point out the necessary assumptions, in addition to those of multidomain model, needed to be made in order to derive the kidney model.
Then, we give the steady state model used in Alan Weinstein's recent papers.
This is in hope that we have a clear overall picture which will ease us in the analysis of the model.

\section{Model derivation}

Consider a one-dimensional domain $\Omega = (-2,7)$ which represents the radial coordinate of the kidney tissue from the superficial to the deeper layers; here, we assume that there is no variation across the kidney tissue of the same depth.
Specifically, $(-2,0)$ represents medullary ray, and $[0,7)$ represents renal medulla.
The domain is multiphasic in the sense that multiple radially aligned compartments of the kidney --- which include the lumina, tubular cells and their intercellular space (also called lateral interspace) of different tubular segments (except convoluted tubules and connecting tubules), interstitia of medulla together with medullary ray, and vasa recta --- are formally described within the same region, $\Omega$.

Anatomically speaking, the renal cortex also share the same region $(-2,0)$ as medullary ray.
However, since the geometry of the convoluted tubules and connecting tubules, which are contained in the renal cortex, are difficult to describe, the spatial variations within the interstitium and the capillary plexus of the renal cortex are ignored.
Therefore, we treat the tubules within the cortex as separated systems with their own one-dimensional domain which are coupled to components in $\Omega$ via their boundary and the cortical interstitium.

\subsection{The renal medulla and medullary ray}

We will use the label
% $k=\pm(3N+1)$ the ascending and the descending vasa recta respectively and 
$k=0$ for the interstitium of medullary ray and the renal medulla.
It is important to note that, in this model, we do not make a distinction between the actual interstitium and the capillary plexus present in medullary ray (and the renal cortex.)
Suppose there are $N$ distinct types of nephrons.
For each type of nephron there are $2$ components contained in $\Omega$: the descending and the ascending parts, each of which are divided into the lumen, the intracellular space (ICS), and the lateral intercellular space (LIS).
For the $n^{\mathrm{th}}$ type nephron, $1\leq n\leq N$, we label by $k=\pm n$, $k=\pm(N+n)$, and $k=\pm(2N+n)$ the descending (positive) and the ascending (negative) parts of the lumen, the ICS, and the LIS respectively.
Similarly, for $P$ distinct types of vasa recta, the capillaries of the renal medulla running in opposition, we label by $k=\pm(3N+j)$, for $j=1,\dots,P$, the descending and the ascending vasa recta.
Finally, there are 3 distinct types of collecting tubular cells: principal cell, $\alpha$-intercalted cell, and $\beta$-intercalted cell
The functions of the two additional cell types are urine acidification by $\alpha$-intercalated cell and urine alkalinization by $\beta$-intercalated cell, and hence the necessity of assigning a distinct compartment for each of them.
We assign $k=K,K+1,K+2,K+3,K+4$, where $K:=3N+P+1$, the lumen, the ICS of principal, $\alpha$- and $\beta$-intercalted cells, and the LIS of the collecting tubules respectively.

We describe the occupied volume of each $k^{\mathrm{th}}$ compartment, $-3N-P\leq k\leq 3N+P+5=K+4$, in $\Omega$ by the volume per unit depth $\alpha_k:\Omega_k\times [0,\tau)\to \mathbb{R}_+$, where $0<\tau\leq \infty$ and $\Omega_k\subset\Omega$ is an open interval with 
\begin{enumerate}[label=(\roman*)]
    \item $\Omega_0=\Omega$,
    \item $\Omega_{N'+n} = \Omega_{2N'+n}\subset \Omega_n$, and $\inf\Omega_n = \inf\Omega_{N'+n}\leq 0$ for $n=\pm 1,\dots\pm N$ where we denote $N'+n := \mathrm{sign}(n)N+n$ and $2N'+n := 2\, \mathrm{sign}(n)N+n$,
    \item $\inf \Omega_{3N+j} = \inf \Omega_{-3N-j}=0$ for $j=1,\dots,P$, 
    \item $\sup\Omega_k = \sup\Omega_{-k}$ for $k=1,\dots,N,3N+1,\dots,3N+P$,
    \item $\Omega_K = \Omega_{K+1} = \Omega_{K+4} = \Omega$, and $\Omega_{K+2}=\Omega_{K+3}=(-2,0)$.
\end{enumerate}
We can readily see from (i) that the interstitium is defined everywhere.
In (ii), the sets $\Omega_n\setminus \Omega_{N'+n}$ contains the thin loops of Henle, the ICS and the LIS of which are not defined.
Further, in (iii), each loop of vasa recta begins at and returns to the cortex-medullary junction at $x=0$.
Moreover, we must have turning points of the loops of Henle and vasa recta at the end and the beginning of their corresponding descending and the ascending compartments.
Finally, the last condition tell us that $\alpha$- and $\beta$-intercalted cells can be found only in the cortical collecting duct (CCD), which is in medullary ray.

% We require that, by setting $\alpha_k=0$ in $\Omega\setminus\Omega_k$, we must have
% \begin{equation}\label{eq:const_vol}
%     \sum_{k=-3N-P}^{3N+P+3} \alpha_k = \alpha
% \end{equation}
%     for a given volume density function $\alpha:\overline{\Omega}\to\mathbb{R}_+$.
% That is, we assume that the total volume of the medullary tissue is constant.
% Note that, in the steady state model of kidney, $\alpha_k$ is only dependent on the depth $x\in \Omega_k$.
% Therefore, we can effectively treat $\alpha_k$ as the dimensionless volume of the compartment $k$.

For time-dependent model and $n=\pm 1,\dots,\pm N$, we have the equations for the tubular compartments:
\begin{alignat}{2}
    \frac{\partial \alpha_n}{\partial t} + \frac{\partial}{\partial x}(\alpha_nu_n) &= -\gamma_1^nw_{1}^n-\gamma_2^nw_{2}^n\quad \quad \quad \ \ &&\text{in}\quad \Omega_{n},\\
    \frac{\partial \alpha_{N'+n}}{\partial t} &= \gamma_1^nw_{1}^n-\beta_1^nv_1^n-\lambda_n\ell_0^n\quad &&\text{in}\quad \Omega_{N'+n},\\
    \frac{\partial \alpha_{2N'+n}}{\partial t} &= \gamma_2^nw_{2}^n-\beta_2^nv_2^n+\lambda_n\ell_0^n\quad &&\text{in}\quad \Omega_{2N'+n}=\Omega_{N'+n},\\
    \gamma_m^nw_m^n &= \beta_m^nv_m^n,\quad m=1,2,\quad \ &&\text{in}\quad \Omega_n\setminus\Omega_{N'+n}.
    % \alpha_n = \alpha_{-n},&\quad u_n = -u_{-n}\qquad\qquad\quad \ \text{on}\ \ \sup\Omega_n
\end{alignat}
% and the ascending tubular compartments:
% \begin{align}
%     \frac{\partial \alpha_{-n}}{\partial t} + \frac{\partial}{\partial x}(\alpha_{-n}u_{-n}) &= -\gamma_{-n}(w_{1}^{-n}+w_{2}^{-n})\quad \quad \quad \ \ \text{in}\quad \Omega_{-n},\\
%     \frac{\partial \alpha_{-N-n}}{\partial t} &= \gamma_{-n}w_{1}^{-n}-\ell_0^{-n}-\beta_{-n}v_1^{-n}\quad \text{in}\quad \Omega_{-N-n},\\
%     \frac{\partial \alpha_{-2N-n}}{\partial t} &= \gamma_{-n}w_{2}^{-n}+\ell_0^{-n}-\beta_{-n}v_2^{-n}\quad \text{in}\quad \Omega_{-2N-n},
% \end{align}
Here, $u_n:\overline{\Omega_n}\times[0,\tau)\to \mathbb{R}$;
$\gamma_m^n,\beta_m^n:\Omega_n\to \mathbb{R}_+$, $m=1,2$ represent the area per unit depth at the luminal membrane the basement membrane of the ICS ($m=1$) and the LIS ($m=2$), and $\lambda_n:\Omega_{N'+n}\to \mathbb{R}_+$ is that of the cell membrane between the ICS and the LIS;
$w^n_m,v^n_m:\Omega_n\times [0,\tau)\to \mathbb{R}$ are the water flux (per unit depth) from the lumen and into the interstitium respectively with the subscript $m=1,2$ represents into or from the ICS and the LIS; and $\ell_0^n:\Omega_{N'+n}\times [0,\tau)\to \mathbb{R}$ is the water flux from the ICS into LIS.
We will give the equations for $u_m^n,v_m^n,w_m^n,\ell_0^n$ later.

Similarly, for the collecting tubules ($k=K,\dots,K+4$):
\begin{alignat}{2}
    \frac{\partial \alpha_K}{\partial t} + &\frac{\partial}{\partial x}(\alpha_Ku_K) = -\sum_{m=1}^4\gamma_m^Kw_{m}^K\qquad \qquad \ &&\text{in}\quad \Omega,\\
    \frac{\partial \alpha_{K+m}}{\partial t} &= \gamma_m^Kw_{m}^K-\beta_m^Kv_m^K-\lambda_m^K\ell_0^{K,m},\quad m=1,2,3,\qquad &&\text{in}\quad \Omega_{K+m},\\
    \frac{\partial \alpha_{K+4}}{\partial t} &= \gamma_4^Kw_{4}^K-\beta_4^Kv_4^K+\sum_{m=1}^3\lambda_m^K\ell_0^{K,m}\qquad \quad &&\text{in}\quad \Omega,
\end{alignat}
where $u_K:\overline{\Omega}\times[0,\tau)\to \mathbb{R}$,  $w_m^K,v_m^K,\ell_0^{K,m}:\Omega_{K+m}\times[0,\tau)\to \mathbb{R}$ and $\gamma_m^K,\beta_m^K,\lambda_m^K:\Omega_{K+m}\to \mathbb{R}_+$ are interpreted the same as previously.
By convention, these functions are set to $0$ outside their defined domain.

For vasa recta ($k=3N'+j:=3\,\mathrm{sign}(j)N+j$, $j=\pm 1,\dots,\pm P$) and the interstitium ($k=0$), we have:
\begin{align}
    \frac{\partial \alpha_{3N'+j}}{\partial t} + \frac{\partial}{\partial x}(\alpha_{3N'+j}u_{3N'+j}) = -\eta_{j}\omega_{j}\qquad &\text{in}\quad \Omega_{3N'+j},\\
    \frac{\partial \alpha_0}{\partial t} + \frac{\partial}{\partial x}(\alpha_0u_0) = 
        \sum_{\substack{1\leq|n|\leq N\\\text{or } n=K}}\sum_m\beta_m^nv_{m}^n+\sum_{|j|\leq P+1}\eta_j\omega_j \quad &\text{in}\quad \Omega.\label{eq:s_water}
    % \alpha_n = \alpha_{-n},\quad u_n = -u_{-n}\qquad\qquad\quad \ &\text{on}\ \ \sup\Omega_n
\end{align}
Here, $u_0:\overline{\Omega}\times(0,\tau]\to \mathbb{R}$ and $u_{3N'+j}:\overline{\Omega_{3N'+j}}\times(0,\tau]\to \mathbb{R}$ are the water flow velocity, $\omega_j:\Omega_{3N'+j}\times[0,\tau)\to \mathbb{R}$, $j=\pm 1,\dots,\pm P$, is the water flux from the vessel.
Likewise, $\omega_0,\omega_{\pm(P+1)}:(-2,0)\times[0,\tau)\to \mathbb{R}$ the cortex ($j=0$) into the interstitium, and the input and output ($j=\pm(P+1)$) \textit{into} the capillary plexus of medullary ray.
Again, for convenience, these are set to $0$ outside their domain.
% In Alan Weinstein's steady state model, $\eta_{P+1}\omega_{P+1}$ is given and $\eta_{-P-1}\omega_{-P-1}$ is computed so that the balance equation is satisfied, but more generally, we can have them depending on pressures which we will describe later.
The parameter $\eta_j:\overline{\Omega_{3N'+j}}\to\mathbb{R}_+$, $j=\pm 1,\dots,\pm P$, are the area per unit depth of the wall of vasa recta.
Similarly, $\omega_0,\omega_{\pm(P+1)}:(-2,0)\to \mathbb{R}$ are those of the cortex-medullary ray junction and the vessels supplying the capillary plexus.
% Note also that, in Alan Weinstein's steady state model 
% the left-hand side of (\ref{eq:s_water}) is zero by an assumption that $u_0$ is identically zero in $\Omega$.
% Additionally, it is also assumed that $\omega_0\equiv 0$ in his model, i.e., the interstitia of the cortex and medullary ray are completely separated.

% Further, we have the condition at the turning point of loops of Henle and vasa recta:
% \begin{equation}
%     \alpha_ku_k = -\alpha_{-k}u_{-k},\quad\text{on}\quad\sup\Omega_k,\quad k= 1,\dots,3N+P.
% \end{equation}

Now, let there be $M$ mobile solute species, e.g., \ce{Na+}, \ce{K+}, \ce{Cl-}, glucose, \ce{CO2}, etc., which are labeled by $i=1,\dots,M$.
We denote by $c_i^k:\overline{\Omega_k}\times [0,\tau)\to \mathbb{R}_+$ the concentration of the $i^{\mathrm{th}}$ solute in the $k^{th}$ compartment, and by $a_i^k:=\alpha_kc_i^k$ the solute amount of $i$ in $k$ per unit depth.
We have the equations for the solutes in nephron compartments, with $n=\pm 1,\dots,\pm N$ and $i=1,\dots,M$:
\begin{alignat}{2}
    \frac{\partial }{\partial t}(\alpha_nc_i^n) &= -\frac{\partial f_i^n}{\partial x}-\gamma_1^ng_i^{n,1}-\gamma_2^ng_i^{n,2}+s_i^{n} \qquad \ &&\text{in}\quad \Omega_{n},\\
    \frac{\partial }{\partial t}(\alpha_{N'+n}c_i^{N'+n}) &= \gamma_1^ng_i^{n,1}-\beta_1^nq_i^{n,1}-\lambda_n\ell_i^n+s_i^{N'+n} \quad \ &&\text{in}\quad \Omega_{N'+n},\\
    \frac{\partial }{\partial t}(\alpha_{2N'+n} c_i^{2N'+n}) &= \gamma_2^ng_i^{n,2}-\beta_2^nq_i^{n,2}+\lambda_n\ell_i^n+s_i^{2N'+n}\quad &&\text{in}\quad \Omega_{N'+n},\\
    \gamma_m^ng_i^{n,m} &= \beta_m^nq_i^{n,m},\quad m=1,2, \qquad \qquad \quad \text{in}&&\quad \Omega_n\setminus\Omega_{N'+n},
\end{alignat}
    where $g_i^{n,m},q_i^{n,m}:\Omega_n\times [0,\tau)\to \mathbb{R}$ are the solute flux (per unit depth) from the lumen and into the interstitium with the superscript $m=1,2$ represents into or from the ICS and the LIS respectively, $\ell_i^n:\Omega_{N'+n}\to \mathbb{R}$ is the solute flux from the ICS into LIS, and $s_i^{k}:\Omega_k\times [0,\tau)\to \mathbb{R}$ is the generation of the solute $i$ in the $k^{\mathrm{th}}$ compartment.
Similarly, for collecting tubules, we also have
\begin{alignat}{2}
    \frac{\partial }{\partial t}(\alpha_Kc_i^K) &= -\frac{\partial f_i^K}{\partial x}-\sum_{m=1}^4 \gamma_m^Kg_i^{K,m}+s_i^K \qquad \ &&\text{in}\quad \Omega,\\
    \frac{\partial }{\partial t}(\alpha_{K+m}c_i^{K+m}) &= \gamma_m^Kg_i^{K,m}-\beta_m^Kq_i^{K,m}-\lambda_m^K\ell_i^{K,m}+s_i^{K+m} \quad \ &&\text{in}\quad \Omega,\\
    \frac{\partial }{\partial t}(\alpha_{K+2} c_i^{K+2}) &= \gamma_4^Kg_i^{K,4}-\beta_4^Kq_i^{K,4}+\sum_{m=1}^3\lambda_m^K\ell_i^{K,m}+s_i^{K+4}\quad &&\text{in}\quad \Omega.
\end{alignat}

We will use $i=M$ to represent \ce{CO2} and write $\mathbf{s}_k := (s_1^k,\dots,s_M^k)^\top$ for all tubular compartments $k=\pm 1,\dots,\pm 3N,\, K,\dots,K+4$.
We describe the tubular solute generation by an invertible reaction matrix $R\in \mathbb{Z}^{M\times M}$ and the metabolic generation $\mathbf{r}_k$, whose value is in $\mathbb{R}^M$ so that we have
\begin{equation}\label{eq:gen}
    R \mathbf{s}_k = \alpha_k\mathbf{r}_k.
\end{equation}
The trivial case of $R$ would be $R=I$, the identity matrix, and $\mathbf{r}_k=\mathbf{0}$.
In case that there are buffer reactions, $R$ represents both mass conservation and the reaction kinetics captured by $\mathbf{r}_k$.
For example, when the solutes are \ce{Na+}, \ce{Cl-}, \ce{H+}, \ce{HCO3-}, \ce{H2CO3}, and \ce{CO2}, with a reaction \ce{H+ + HCO3- <=> H2CO3 <=> H2O + CO2} we might have
\begin{equation*}
    \begin{pmatrix}
        1 & 0 & 0 & 0 & 0 & 0\\
        0 & 1 & 0 & 0 & 0 & 0\\
        0 & 0 & 1 & -1& 0 & 0\\
        0 & 0 & 0 & 1 & 1 & 1\\
        0 & 0 & 0 & 0 & 1 & 0\\
        0 & 0 & 0 & 0 & 0 & 1\\
    \end{pmatrix}
    \begin{pmatrix}
        s_{\ce{Na+}}^k \\  s_{\ce{Cl-}}^k \\ s_{\ce{H+}}^k \\ s_{\ce{HCO3-}}^k \\ s_{\ce{H2CO3}}^k \\ s_{\ce{CO2}}^k
    \end{pmatrix}
    = \alpha_k\begin{pmatrix}
        0 \\ 0 \\ 0 \\ m_k \\ k_1^+c_{\ce{HCO3-}}^kc_{\ce{H+}}^k-k_1^-c_{\ce{H2CO3}}^k\\
        k_2^+c_{\ce{H2CO3}}^k - k_2^-c_{\ce{CO2}}^k+m_k
    \end{pmatrix}
\end{equation*}
    where $k_1^+,k_1^-,k_2^+,k_2^-$ are reaction rate constants and $m_k$ is the rate of oxidative metabolism generating \ce{CO2}.
In general, $\mathbf{r}_k$ is a function depending on $\mathbf{c}_k:=(c_1^k,\dots,c_M^k)^\top$, and the fluxes of active membrane transports when $k$ is a cellular compartment which generates \ce{CO2} via oxidative metabolism.

Alternatively, we can take an approximation that some reaction is instantaneous, i.e., the reaction is always at the equilibrium.
In this case, $R$ is not invertible.
Specifically, the rank of $R$ is now deficient by the number of the reactions assumed to be instantaneous.
However, the equation (\ref{eq:gen}) can still determine $\mathbf{s}_k$ since $\mathbf{r}_k$ on the right-hand side now supplement the `missing' equation(s).
For instance, if we assumed that the reaction \ce{H+ + HCO3- <=> H2CO3} in the previous example is instantaneous, now we would have
\begin{equation*}
    \begin{pmatrix}
        1 & 0 & 0 & 0 & 0 & 0\\
        0 & 1 & 0 & 0 & 0 & 0\\
        0 & 0 & 1 & -1& 0 & 0\\
        0 & 0 & 0 & 1 & 1 & 1\\
        0 & 0 & 0 & 0 & 0 & 0\\
        0 & 0 & 0 & 0 & 0 & 1\\
    \end{pmatrix}
    \begin{pmatrix}
        s_{\ce{Na+}}^k \\  s_{\ce{Cl-}}^k \\ s_{\ce{H+}}^k \\ s_{\ce{HCO3-}}^k \\ s_{\ce{H2CO3}}^k \\ s_{\ce{CO2}}^k
    \end{pmatrix}
    = \alpha_k\begin{pmatrix}
        0 \\ 0 \\ 0 \\ m_k \\ k_1^+c_{\ce{HCO3-}}^kc_{\ce{H+}}^k-k_1^-c_{\ce{H2CO3}}^k\\
        k_2^+c_{\ce{H2CO3}}^k - k_2^-c_{\ce{CO2}}^k+m_k
    \end{pmatrix}
\end{equation*}
    from which we have $c^k_{\ce{H2CO3}}/(c^k_{\ce{HCO3-}}c^k_{\ce{H+}}) = k_1^+/k_1^-$ which, together with the balance equation, determines $s_{\ce{H2CO3}}^k$.

For vasa recta ($k=3N'+j,\,j=\pm 1,\dots,\pm P$) and the interstitium ($k=0$) --- recall that we treat the interstitium and the plasma of the capillary plexus of medullary ray as a single compartment, there are additional solute species, namely hemoglobin buffer species which we label by $i=M+1,\dots,M+B$.
These additional solute species provide additional buffering for \ce{H+}, and \ce{CO2}.
For the hemoglobin buffer species, we have $c_i^k:\overline{\Omega_k}\times [0,\tau)$, $i=M+1,\dots,M+B$, for $k=3N'+j$ and with $c_i^0:(-2,0)\times [0,\tau)\to \mathbb{R}_+$, be the concentration with respect to the \textit{total blood volume} per unit depth, not just plasma.
In other words, we have $a_i^k = \tilde{\alpha}_kc_i^k$ for $i=M+1,\dots,M+B$ where $\tilde{\alpha}_k:=\alpha_k+\alpha_k^\mathrm{b}$ with $\alpha_k^\mathrm{b}:\overline{\Omega_k}\to \mathbb{R}_+$, $k=3N'+j$, and $\alpha^\mathrm{b}_0:(-2,0)\to \mathbb{R}_+$ being the volume distribution of the red blood cells.
We have the solute equations, with $i=1,\dots,M$, for vasa recta:
\begin{align}
    \frac{\partial }{\partial t}(\alpha_{3N'+j}c_i^{3N'+j}) = - \frac{\partial f_i^{3N'+j}}{\partial x} -\eta_{j}\vartheta_i^j+s_i^{3N'+j}\qquad &\text{in}\quad \Omega_{3N'+j}
    % \\
    % \frac{\partial}{\partial t}(\alpha_0 c_i^0) = - \frac{\partial f_i^0}{\partial x} + 
    %     \sum_{\substack{1\leq|n|\leq N\\\text{or } n=K}}\beta_n(q_{1}^n+q_{2}^n)+\sum_{0\leq |j|\leq P}\eta_j\vartheta_i^j\quad &\text{in}\quad \Omega.
    % \alpha_n = \alpha_{-n},\quad u_n = -u_{-n}\qquad\qquad\quad \ &\text{on}\ \ \sup\Omega_n
\end{align}
    and for $i=M+1,\dots,M+B$:
\begin{equation}\label{eq:vr_sol}
    \frac{\partial }{\partial t}(\tilde{\alpha}_{3N'+j}c_i^{3N'+j}) = - \frac{\partial f_i^{3N'+j}}{\partial x} +s_i^{3N'+j}\qquad \text{in}\quad \Omega_{3N'+j}.
\end{equation}
For the interstitium, we have the equations for $i=1,\dots,M$:
\begin{equation}\label{eq:int_sol}
    \frac{\partial}{\partial t}(\alpha_0 c_i^0) = - \frac{\partial f_i^0}{\partial x} + 
        \sum_{\substack{1\leq|n|\leq N\\\text{or } n=K}}\sum_m\beta_m^nq_m^n+\sum_{|j|\leq P+1}\eta_j\vartheta_i^j+s_i^0\quad \text{in}\quad \Omega,
\end{equation}
    and for $i=M+1,\dots,M+B$:
\begin{equation}
    \frac{\partial}{\partial t}(\tilde{\alpha}_0 c_i^0) = 
    % - \frac{\partial f_i^0}{\partial x} + 
    \eta_{P+1}\vartheta_i^{P+1}+\eta_{-P-1}\vartheta_i^{-P-1}+s_i^0\quad \text{in}\quad (-2,0).
\end{equation}
Here, $\vartheta_i^j:\Omega_{3N'+j}\times [0,\tau)\to \mathbb{R}$, $j=\pm 1,\dots,\pm P$, and $\vartheta_i^j:(0,2)\times [0,\tau)\to \mathbb{R}$, $j=0,\pm(P+1)$, are the solute fluxes from vasa recta ($j=\pm 1,\dots,\pm P$) and the renal cortex ($j=0$) into the interstitium, and the vascular input and output ($j=\pm(P+1)$) \textit{into} the capillary plexus of medullary ray.
In the equation (\ref{eq:int_sol}), we also set $\vartheta_i^j=0$ in $\Omega\setminus \Omega_{3N'+j}$ for $j=\pm 1,\dots,\pm P$, and $q_1^n,q_2^n=0$ in $\Omega\setminus\Omega_n$; similarly, $\vartheta_i^{\pm(P+1)} = 0$ in $[0,7)$.
Note that it is assumed $f_i^0\equiv 0$ and $\vartheta_i^0\equiv 0$ in Alan Weinstein's model.
Moreover, since there is no need to explicitly describe $\vartheta_i^j$, $j=\pm(P+1)$, in the steady state model, $\vartheta_i^{P+1}$ is given and $\vartheta_i^{-P-1}$ is a model unknown in such a case; we need to have equations for these for the time-dependent model.
Note that, since we need to account for the total blood volume for the hemoglobin buffer species, the axial flux $f_i^{3N'+j}$ in the equation (\ref{eq:vr_sol}) are slightly different from that when $i=1,\dots,M$.
That is, the convection need to account for the total blood flow instead of the plasma flow.

The reaction terms $s_i^k$, $k=0,\pm(3N+1),\dots,\pm(3N+P)$, also have similar equations as (\ref{eq:gen}), but now with $\mathbf{s}_k:=(s_1^k,\dots,s_M^k,s_{M+1}^k,\dots,s_{M+B}^k)$, $R$ is replaced by another non-invertible matrix $R'\in \mathbb{R}^{(M+B)\times (M+B)}$ and $\mathbf{r}_k$ has value in $\mathbb{R}^{M+B}$, i.e.,
\begin{equation}
    R'\mathbf{s}_k=\mathrm{diag}(\underbrace{\alpha_k,...}_{M},\underbrace{\tilde{\alpha}_k,\dots}_B)\mathbf{r}_k,\quad k=0,\pm(3N+1),\dots,\pm(3N+P)
\end{equation}
with $\mathbf{r}_k$ depending on $\mathbf{c}_k:=(c_1^k,\dots,c_{M+B}^k)^\top$ which provides the `missing' equations, as before.

Now, we describe the water flow velocity, $u_k$ and the solute flow within each compartment $f_i^k$.
In Alan Weinstein's steady state model, there is no axial flow in the interstitium ($k=0$) --- that is $u_0\equiv 0$ and $f_i^0\equiv 0$ for all $i$.
For $k=\pm 1,\dots,\pm N,\pm(3N+1),\dots,\pm(3N+P),K$, 
the water flow is given by Poisseuille's equations
\begin{align}
    \frac{\rho_ku_k}{\alpha_k} &= -\frac{\partial p}{\partial x},\quad k=\pm 1,\dots,\pm N, K,\\
    \frac{\rho_ku_k}{\tilde{\alpha}_k} &= -\frac{\partial p}{\partial x},\quad k=\pm(3N+1),\dots,\pm(3N+P),
\end{align}
where $\rho_k/\alpha_k$ and $\rho_k/\tilde{\alpha}_k$ are the hydraulic resistivity with $\rho_k$ constant and $p_k$ is the hydrostatic pressure, and the solute flows $f_i^k$, $i=1,\dots,M$, are assumed to be purely convective, i.e., $f_i^k = \alpha_ku_kc_i^k$.
In general, $f_i^k$ can be generalized into Nernst-Planck equation:
\begin{equation}
    f_i^k = -D_i^k\left( \frac{\partial c_i^k}{\partial x}+\frac{z_iFc_i^k}{RT} \frac{\partial\phi_k}{\partial x}\right) + \alpha_k u_kc_i^k,\quad i=1,\dots,M,
    % \quad k\neq \pm (2N+1),\dots,\pm 3N
\end{equation}
where $D_i^k$ is the diffusion coefficient, $z_i$ is the valence of the solute $i$, $F/RT$ is a constant, $\phi_k$ is the electrical potential in compartment $k$.
% and the water flow velocity $u_k$
% is determined by the compartmental pressure field and driven by electrostatic force:
% \begin{gather}
%     \frac{\rho_ku_k}{\alpha_k} = -\frac{\partial p}{\partial x} - \sum_{i=1}^{M'}z_iFc_i^k\frac{\partial \phi_k}{\partial x},\\
%     % \quad k\neq \pm (2N+1),\dots,\pm 3N,\\
%     M'=\begin{cases}
%         M&\quad \text{if}\quad k=\pm 1,\dots,\pm N,K,K+1,K+2,\\
%         M+B &\quad \text{if}\quad k=0,\pm(3N+1),\dots,\pm(3N+P).
%     \end{cases}\nonumber
% \end{gather}
% where $\tilde{ p }_k:= p_k-RTa_k/\alpha_k$ is the compartmental pressure with $a_k$ being the amount per unit depth of immobile solute.
% Finally, the solute and water flows at the turning points of loops of Henle and vasa recta must match, i.e.,
% \begin{align}
    %     \alpha_ku_k = -\alpha_{-k}u_{-k},\quad\text{on}\quad\sup\Omega_k,\quad k= 1,\dots N,3N+1,\dots,3N+P,\\
    %     f_i^k = -f_i^{-k},\quad\text{on}\quad\sup\Omega_k,\quad k= 1,\dots N,3N+1,\dots,3N+P.
    % \end{align}
For the hemoglobin buffer species ($i=M+1,\dots,M+B$), the flows are given (only in vasa recta) by
\begin{equation}
    f_i^k = \tilde{\alpha}_ku_kc_i^k
        \quad\text{in}\quad \Omega_k,\quad\text{for}\quad k=\pm(3N+1),\dots,\pm(3N+P),
\end{equation}
% \begin{equation}
%     f_i^k = \tilde{\alpha}_ku_kc_i^k,\begin{cases}
%         \quad\text{in}\quad \Omega_k,&\quad\text{for}\quad k=\pm(3N+1),\dots,\pm(3N+P),\\
%         \quad\text{in}\quad (-2,0),&\quad\text{for}\quad k=0,\\
%     \end{cases}
% \end{equation}
where $\tilde{\alpha}_ku_k$ is the \textit{blood flow} which must satisfy the incompressibility of the red blood cell, $\partial (\alpha^\mathrm{b}_k u_k)/\partial x = 0$, or equivalently:
\begin{equation}
    \frac{\partial}{\partial x}\left( \tilde{\alpha}_k u_k\right) = \frac{\partial}{\partial x}\left( \alpha_k u_k \right).
\end{equation}
    
To determine the electrical potential $\phi_k$ and the hydrostatic pressure in each compartment, we have an electroneutrality approximation:
\begin{equation}
    0=z_0^k Fa_k+\sum_{i=1}^{M'}z_iF \alpha_k c_i^k,\quad k=-3N-P,\dots,K+2,
\end{equation}
and the pressure balance between compartments are described by compliances $\nu_k$.
For the luminal compartments ($k=\pm 1,\dots,\pm N,\pm (3N+1),\dots,\pm(3N+p), K$), we have:
%  $k=\pm 1,\dots,\pm 2N,\pm(3N+1),\dots,\pm(3N+P),K,K+1$:
\begin{align}
    \nu_k(p_k - p_0) &= \frac{\alpha_k}{\alpha_k^0}-1.
    % \\
    % p_{N'+n} &= p_n,\\
    % \nu_{2N'+n}(p_{2N'+n} - p_{N'+n}) &= \frac{\alpha_{2N'+n}}{\alpha_{2N'+n}^0}-1,
\end{align}
and for the ICS and LCS, with $n=\pm 1,\dots,\pm N$:
\begin{align}
    p_{N'+n} &= p_n,\\
    \nu_{2N'+n}(p_{2N'+n} - p_{N'+n}) &= \frac{\alpha_{2N'+n}}{\alpha_{2N'+n}^0}-1,
\end{align}
where $\alpha_k^0$ are the baseline volumes in which the pressure on both sides are equal.
Note that the interstitial pressure is determined so that the equation (\ref{eq:const_vol}) is satisfied.
% For the ICS and ECS compartments, we set, for $n=\pm 1,\dots,\pm N$:
% \begin{gather}
%     p_{N'+n} = p_n,\quad p_K = p_{K+1}.
%     % p_{2N'+n} = p_0 = p_{K+2}.
% \end{gather}
% Note that, in Alan Weinstein's model, the hydrostatic pressures of vasa recta and interstitium are computed from the balance equation, since the volume densities of vasa recta and interstitium are fixed.
% These conditions are functionally equivalent to our equation (\ref{eq:const_vol}).

Now, we describe the boundary conditions for the time-dependent model which include those of hydrostatic pressures, water flow velocities, and solute flows.
The water and solute flows at the beginning of the descending tubules, including the collecting tubules, are determined by those at the ending point of the proximal convoluted tubule (PCT) or the connecting tubules (CNT) in the renal cortex:
\begin{equation}
    \left.
    \begin{aligned}
        \alpha_n u_n &= \bar{\alpha}_n(L_n,\cdot)\bar{u}_n(L_n,\cdot),\\
        f_i^n &= \bar{f}_i^n(L_n,\cdot),\\
        % p_n &= \bar{p}_n(L_n,\cdot),
    \end{aligned}\right\}
    \quad\text{on}\quad \inf\Omega_n,\quad n=1,\dots,N,K,
\end{equation}
where $\bar{\alpha}_n,\bar{u}_n,\bar{f}_i^n,\bar{p}_n:[0,L_n]\times [0,\tau)\to \mathbb{R}_+$ are the cross-sectional area of the PCT ($n=1,\dots,N$) and the CNT ($n=K$) of length $L_n>0$, the water flow velocity, the solute flow, and the hydrostatic pressure inside the lumen respectively.
For the descending vasa recta, the water and solute flows are given at the begining:
\begin{equation}
    \left.
        \begin{aligned}
            \alpha_k u_k &= \bar{u}_k,\\
            f_i^k &= \bar{f}_i^k,\\
            % p_k &= \bar{p}_k,
        \end{aligned}\right\}
        \quad\text{on}\quad \inf\Omega_k = 0,\quad k=3N+1,\dots,3N+P,
\end{equation}
where $\bar{\alpha}_k,\bar{u}_k,\bar{f}_i^k,\bar{p}_k:[0,\tau)\to \mathbb{R}_+$ are given.
Further, the solute and water flows and hydrostatic pressure at the turning points of loops of Henle and vasa recta must match, i.e.,
\begin{equation}
\left.
    \begin{aligned}
        \alpha_ku_k &= -\alpha_{-k}u_{-k},\\
        f_i^k &= -f_i^{-k},\\
        % p_k &= p_{-k}
    \end{aligned}\right\}
    \quad\text{on}\quad\sup\Omega_k,\quad k= 1,\dots N,3N+1,\dots,3N+P.
\end{equation}
% At the beginning of the collecting tubules, we have the water and solute flows and the hydrostatic pressure taken from those at the end of the connecting tubule (CNT) contained in the renal cortex, i.e.,
% \begin{equation}
%     \left.
%     \begin{aligned}
%         \alpha_K u_K &= \bar{\alpha}_K(L_K,\cdot)\bar{u}_K(L_K,\cdot),\\
%         f_i^K &= \bar{f}_i^K(L_K,\cdot),\\
%         p_K &= \bar{p}_K(L_K,\cdot),
%     \end{aligned}\right\}
%     \quad\text{on}\quad \inf\Omega_n,\quad n=1,\dots,N,
% \end{equation}
% where $\bar{\alpha}_K,\bar{u}_K,\bar{f}_i^K,\bar{p}_K:[0,L_K]\times [0,\tau)\to \mathbb{R}_+$ are the cross-sectional area of the CNT of length $L_K>0$, the water flow velocity, the solute flow, and the hydrostatic pressure inside the lumen respectively.
{\color{red}(Do the above boundary conditions, together with the continuity at the turning point of the initial conditions of concentrations, electrical potentials, and hydrostatic pressures, imply continuity at the turning point for all time?
Would it be redundant to include such continuity conditions into the boundary conditions?)}
Finally, we have no-flux boundary conditions for the interstitium:
\begin{gather}
    u_0(-2,\cdot) = u_0(7,\cdot)=0,\\
    f_i^0(-2,\cdot) = f_i^0(7,\cdot) = 0.
\end{gather}
We are now left with the description of the renal cortex and its coupling with the renal medulla and medullary ray.

\subsection{The renal cortex}

Within the cortex, we have the cortical interstitium, the proximal convoluted tubules (PCT), the distal convoluted tubules (DCT), and the connecting tubules (CNT).
We use the same convention to label these compartments.
We have $k=0$ for the cortical interstitium together with the plasma of the capillary plexus.
However, unlike the medullary ray, we shall treat this compartment as a homogeneous compartment with no spatial variation.
For $N$ types of nephrons, the lumen of PCT are labeled by $k=1,\dots,N$ and that of the DCT by $k=-1,\dots,-N$, with $k=\pm n$ belongs to the $n^{\mathrm{th}}$ type nephron.
As before, we also have the ICS and the LIS labeled by $k=\pm(N+1),\dots,\pm 3N$.
Lastly, the CNT, which receives flows from all types of nephrons is labeled by $k=K,K+1,K+2$, corresponding to the lumen, the ICS, and the LIS respectively.

We describe the volume of the cortical interstitium by $\bar{\alpha}_0:[0,\tau)\to \mathbb{R}_+$, and the cross-sectional area of each tubular components by $\bar{\alpha}_{n},\bar{\alpha}_{N'+n},\bar{\alpha}_{2N'+n}:(0,L_n)\to \mathbb{R}_+$, for $n=\pm 1,\dots,\pm N$ and $\bar{\alpha}_K,\bar{\alpha}_{K+1},\bar{\alpha}_{K+2}:(0,L_K)\to \mathbb{R}_+$, where $L_n>0$, $n=\pm 1,\dots,\pm N,K$ are the length of the PCT, the DCT, and the CNT.
We have equations for the compartment volumes, for $n=\pm 1,\dots,\pm N$:
\begin{align}
    \frac{\partial \bar{\alpha}_n}{\partial t} + \frac{\partial}{\partial x}(\bar{\alpha}_n\bar{u}_n) &= -\bar{\gamma}_n(\bar{w}_{1}^n+\bar{w}_{2}^n)\quad \quad \quad \ \ \text{in}\quad (0,L_n),\\
    \frac{\partial \bar{\alpha}_{N'+n}}{\partial t} &= \bar{\gamma}_{n}\bar{w}_{1}^n-\bar{\beta}_n\bar{v}_1^n-\bar{\lambda}_n\bar{\ell}_0^n\quad \text{in}\quad (0,L_n),\\
    \frac{\partial \bar{\alpha}_{2N'+n}}{\partial t} &= \bar{\gamma}_{n}\bar{w}_{2}^n-\bar{\beta}_n\bar{v}_2^n+\bar{\lambda}_n\bar{\ell}_0^n\quad \text{in}\quad (0,L_n),\\
    \frac{\partial \bar{\alpha}_K}{\partial t} + \frac{\partial}{\partial x}(\bar{\alpha}_K\bar{u}_K) &= -\bar{\gamma}_K(\bar{w}_{1}^K+\bar{w}_{2}^K)\qquad \quad \ \ \ \, \text{in}\quad (0,L_K),\\
    \frac{\partial \bar{\alpha}_{K+1}}{\partial t} &= \bar{\gamma}_{n}\bar{w}_{1}^K-\bar{\beta}_K\bar{v}_1^K-\bar{\lambda}_K\bar{\ell}_0^K\quad \text{in}\quad (0,L_K),\\
    \frac{\partial \bar{\alpha}_{K+2}}{\partial t} &= \bar{\gamma}_{n}\bar{w}_{2}^K-\bar{\beta}_K\bar{v}_2^K+\bar{\lambda}_K\bar{\ell}_0^K\quad \text{in}\quad (0,L_K),\\
    \frac{d \bar{\alpha}_0}{d t} = 
        \sum_{\substack{1\leq|n|\leq N\\\text{or } n=K}}\int_0^{L_n}&\bar{\beta}_n(\bar{v}_{1}^n+\bar{v}_{2}^n)\,d\xi+\eta_+\omega_++\eta_-\omega_--\eta_0\omega_0.
\end{align}

Further, we have solutes $i=1,\dots,M$ for the tubular compartments and $i=1,\dots,M+B$ for the cortical interstitium.
The concentrations of the solutes $i=1,\dots,M$ in the tubular components are given below:
\begin{align}
    \frac{\partial }{\partial t}(\bar{\alpha}_n\bar{c}_i^n) &= -\frac{\partial \bar{f}_i^n}{\partial \xi}-\bar{\gamma}_n(\bar{g}_i^{n,1}+\bar{g}_i^{n,2})+\bar{s}_i^{n} \qquad \   \text{in}\quad (0,L_n),\\
    \frac{\partial }{\partial t}(\bar{\alpha}_{N'+n}\bar{c}_i^{N'+n}) &= \bar{\gamma}_{n}\bar{g}_i^{n,1}-\bar{\beta}_n\bar{q}_i^{n,1}-\bar{\lambda}_n\bar{\ell}_i^n+\bar{s}_i^{N'+n} \quad \, \text{in}\quad (0,L_n),\\
    \frac{\partial }{\partial t}(\bar{\alpha}_{2N'+n} \bar{c}_i^{2N'+n}) &= \bar{\gamma}_{n}\bar{g}_i^{n,2}-\bar{\beta}_n\bar{q}_i^{n,2}+\bar{\lambda}_n\bar{\ell}_i^n+\bar{s}_i^{2N'+n}\ \ \text{in}\quad (0,L_n),\\
    \frac{\partial }{\partial t}(\bar{\alpha}_K\bar{c}_i^K) &= -\frac{\partial \bar{f}_i^K}{\partial \xi}-\bar{\gamma}_K(\bar{g}_i^{K,1}+\bar{g}_i^{K,2})+\bar{s}_i^{n} \qquad  \text{in}\quad (0,L_K),\\
    \frac{\partial }{\partial t}(\bar{\alpha}_{K+1}\bar{c}_i^{K+1}) &= \bar{\gamma}_{n}\bar{g}_i^{K,1}-\bar{\beta}_K\bar{q}_i^{K,1}-\bar{\lambda}_K\bar{\ell}_i^K+\bar{s}_i^{K+1} \quad  \text{in}\quad (0,L_K),\\
    \frac{\partial }{\partial t}(\bar{\alpha}_{K+2} \bar{c}_i^{K+2}) &= \bar{\gamma}_{n}\bar{g}_i^{K,2}-\bar{\beta}_K\bar{q}_i^{n,2}+\bar{\lambda}_K\bar{\ell}_i^K+\bar{s}_i^{K+2}\quad \,  \text{in}\quad (0,L_K),
\end{align}
and we have the solute generation
\begin{align}
    R\bar{\mathbf{s}}_k = \bar{\alpha}_k \mathbf{r}_k.
\end{align}
For the interstitial concentration when $i=1,\dots,M$ we have
\begin{equation}
    \frac{d}{d t}(\bar{\alpha}_0 \bar{c}_i^0) =  
        \sum_{\substack{1\leq|n|\leq N\\\text{or } n=K}}\int_0^{L_n}\bar{\beta}_n(\bar{q}_{1}^n+\bar{q}_{2}^n)\,d\xi+\eta_+\vartheta_i^+ + \eta_-\vartheta_i^- -\eta_0\vartheta_i^0+\bar{s}_i^0.
\end{equation}
For the hemoglobin buffer species, $i=M+1,\dots,M+B$, we have
\begin{equation}
    \frac{d}{d t}(\tilde{\bar{\alpha}}_0 \bar{c}_i^0) = 
    % - \frac{\partial f_i^0}{\partial x} + 
    \eta_+\vartheta_i^++\eta_-\vartheta_i^-+\bar{s}_i^0.
\end{equation}
The solute generations in the cortical interstitium and capillary plexus are given by
\begin{equation}
    R'\bar{\mathbf{s}}_0 = \mathrm{diag}(\underbrace{\bar{\alpha}_0,...}_{M},\underbrace{\tilde{\bar{\alpha}}_0,\dots}_B)\mathbf{r}_0.
\end{equation}

We also have the electroneutrality approximation and the compliance of the tubular structures determining the hydrostatic pressure in each compartment.

The model renal cortex is completed with three conditions of the single nephron glomerular filtration rate (SNGFR), which gives the flow at the beginning of the PCT of each nephron, and the CNT end pressure which depends on the terminal flow in the lumen:
\begin{align}
    \bar{\alpha}_n (0,t)\bar{u}_n(0,t) &= \mathrm{SNGFR}\left( \mathbf{c}_{-n}\left( \inf \Omega_{-n},t \right) \right),\quad n=1,\dots,N
\end{align}

\section{Steady state model}



\end{document}