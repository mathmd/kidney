% !TEX program = xelatex
\documentclass{article}

%% ---- Set up paper margin ---- %%
%\usepackage[a4paper,top=1in,bottom=1in,left=1in,right=1in]{geometry}

% use multiple languages
%% ---- allow CJK usage ---- %%
\usepackage[CJKspace]{xeCJK} % this should be called before Polyglossia
\setCJKmainfont{Noto Serif CJK TC}
\setCJKsansfont{Noto Sans CJK TC}
\setCJKmonofont{Noto Sans Mono CJK TC}
% \setCJKmainfont{Noto Serif JP}
% \setCJKsansfont{Noto Sans JP}
% \setCJKmonofont{Noto Sans Mono CJK JP}

%% ---- ตั้งค่าให้ตัดคำภาษาไทย ---- %%
\XeTeXlinebreaklocale "th"
\XeTeXlinebreakskip = 0pt plus 0pt % เพิ่มความกว้างเว้นวรรคให้ความยาวแต่ละบรรทัดเท่ากัน

%% ---- font settings ---- %%
\usepackage{fontspec}
\defaultfontfeatures{Mapping=tex-text} % map LaTeX formating, e.g., ``'', to match the current font
% To change the main font, uncomment one of the below command.
% \setmainfont{TeX Gyre Termes} % Free Times
% \setsansfont{TeX Gyre Heros} % Free Helvetica
% \setmonofont{TeX Gyre Cursor} % Free Courier
\newfontfamily{\thaifont}[Scale=MatchUppercase,Mapping=textext]{Laksaman} % ตั้งฟอนต์หลักภาษาไทย
\newenvironment{thailang}{\thaifont}{} % create environment for Thai language
\usepackage[Latin,Thai]{ucharclasses} % ตั้งค่าให้ใช้ "thailang" environment เฉพาะ string ที่เป็น Unicode ภาษาไทย
\setTransitionTo{Thai}{\begin{thailang}}
\setTransitionFrom{Thai}{\end{thailang}}

%% ---- spacing between lines ---- %%
\usepackage{setspace}
% \singlespacing % default setting
% \onehalfspacing % recommend using this for Thai language

%% ---- using alphabatic language ---- %%
\usepackage{polyglossia}
\setdefaultlanguage{english} % it is preferrable to set English as the main language, since the numeric system is compatible with most LaTeX features such as 'enumerate' and so on
\setotherlanguages{thai}

\AtBeginDocument\captionsthai % allow captions to be in Thai


%% ---- math packages ---- %%
\usepackage{amsmath}
\usepackage{amssymb}
\usepackage{bm} % same functionality as \mathbf{} but for greek letters
\usepackage{diffcoeff}
\usepackage{physics}
\usepackage{mathdots}
% \numberwithin{equation}{section} % equation numbers are formatted as <#Section>.<#eq in the section>

%% ---- define math environment ---- %%
\usepackage{amsthm}
% \newtheorem{definition}{Definition}[section]
\newtheorem{definition}{Definition}
\newtheorem{proposition}[definition]{Proposition}
\newtheorem{theorem}[definition]{Theorem}
\newtheorem{corollary}{Corollary}[definition]
\newtheorem{remark}{Remark}[definition]
\newtheorem{lemma}[definition]{Lemma}
\newtheorem{problem}[definition]{Problem}
% \newtheorem*{problem}{Problem}

%% ---- hyperref settings ---- %%
\usepackage{hyperref}
\usepackage{url}
\usepackage{cite}
\usepackage{xcolor}
\hypersetup{
    colorlinks,
    linkcolor={red!50!black},
    citecolor={green!50!black},
    urlcolor={blue!80!black}
    }

%% ---- misc. packages ---- %%
\usepackage{enumitem} % allow customizing list environments: enumerate, itemize and description.
\usepackage{mhchem} % use chemistry notation
\usepackage{lipsum}
\usepackage{metalogo} % for extended LaTeX logo such as XeTeX
\usepackage{subcaption} % allowing subfigure environment
% \usepackage[section]{placeins} % ensure floats do not go into the next section and allow the use of \FloatBarrier
\usepackage{graphicx} % allow cropping and rotating images
\usepackage{caption}
\usepackage{float} % allow the use of [H] for positioning of tables and figures
\usepackage{tikz-cd}
\restylefloat{table}

%% ---- title, authors, and dates ---- %%
\usepackage{authblk}
\title{A simple time-dependent model of kidney}
\author[1]{Chanoknun Sintavanuruk}
% \affil[1]{Department of Physiology, Faculty of Medicine Siriraj Hospital, Mahidol University}
% \author[2,3]{XXXX XXXX}
% \affil[2]{Department of XXXX, XXXX University}
% \affil[3]{Department of XXXX, XXXX University}
\date{\today}

\begin{document}
\sloppy % ช่วยตัดคำภาษาไทย
\maketitle

\section{Model equations}

\section{Non-dimensionalization}


\section{Numerical method}
Consider a multidomain type model on a (rescaled) spatial domain $\Omega = (0,1)$, on which we have a non-dimensionalized system:
\begin{align}
    \frac{\partial \alpha_k}{\partial t}  + \mathrm{Pe}\frac{\partial}{\partial x}\left( \alpha_k u_k \right) &= - w_k,\\
    \frac{\partial\alpha_0}{\partial t}+\mathrm{Pe}\frac{\partial}{\partial x}\left( \alpha_0 u_0 \right) &=\sum_k w_k,\\
    \frac{\partial}{\partial t}\left( \alpha_k c_i^k \right)&=-\frac{\partial}{\partial x} f_i^k - g_i^k,\\
    \frac{\partial}{\partial t}\left( \alpha_0 c_i^0 \right)&=-\frac{\partial}{\partial x} f_i^0 + \sum_k g_i^k,\\
    \nu_k\left( p_k - p_0 \right) &= \frac{\alpha_k}{\bar{\alpha}_k}-1,\\
    \alpha_0 + \sum_{k} \alpha_k &= \alpha_*,
\end{align}
    where
\begin{align}
    \frac{\rho_\cdot}{\alpha_\cdot}u_\cdot &= -\frac{\partial p_\cdot}{\partial x},\\
    % f_i^\cdot&= -D_i^\cdot\frac{\partial c_i^\cdot}{\partial x}+\mathrm{Pe}\left( \alpha_\cdot u_\cdot c_i^\cdot \right),\\
    f_i^\cdot&= -D_i^\cdot c_i^\cdot\frac{\partial \mu_i^\cdot}{\partial x}+\mathrm{Pe}\left( \alpha_\cdot u_\cdot c_i^\cdot \right),\quad \mu_i^\cdot:=\ln c_i^k \\
    w_k&= \zeta_k\left( \psi_k-\psi_0 \right),\quad\psi_\cdot := p_\cdot - \pi_.,\quad \pi_\cdot := \frac{a_\cdot}{\alpha_\cdot}+\sum_i c_i^\cdot,\\
    g_i^k &= j_i^k+h_i^k,\quad j_i^k :=\gamma_i^k(\mu_i^k-\mu_i^0).
    % := \ln\left( \frac{c_i^k}{c_i^0} \right).
\end{align}
    % where $\mathrm{Pe}$ is the P\'eclet number, $k:\mathrm{D},\mathrm{A},\mathrm{C}$ represents the descending, ascending and collecting tubules of a homogeneous population of nephrons

Let $N\in \mathbb{N}$ be the number of uniformly spaced grids in $(0,1)$, and $\delta x = 1/N$ be the spatial grid size.
Similarly, we denote $\delta t$ as the size of time steps.
We will use the notation $\alpha^{n}_{kl},c_{il}^{kn}, p_{kl}^{n}$ for the discretization of $\alpha_k,c_i^k$ and $p_k$ at the $l$-th spatial grid and time $t = n\delta t$.

We define difference quotient operators:
\begin{equation}
    \Delta_x^+y^n_l:= \frac{y^n_{l+1}-y^n_{l}}{\delta x},\quad \Delta_x^+ y^n_l :=\frac{y^n_l-y^n_{l-1}}{\delta x},\quad \Delta_t y^n_l = \frac{y^n_l - y^{n-1}_l}{\delta t}.
\end{equation}
and an average operator:
\begin{equation}
    Ay_l^n := \frac{y^n_{l+1}+y^n_l}{2}.
\end{equation}

We start with known values of $\alpha^{n-1}_{kl},c_{il}^{k,n-1}, p_{kl}^{n-1}$, $l = 1,\dots,N$.
The first step is to update the unknowns for the next time step $n$.
We have
\begin{gather}
    \Delta_t \alpha_{kl}^n +\mathrm{Pe}(\alpha_{kl}^n u_{kl}^n) = -w_{kl}^{n},\quad w_{kl}^n:=\zeta_k\left( \psi_{kl}^n-\psi_{0l}^n \right)\\
    \psi_{\cdot l}^n:= p_{.l}^n - \pi_{\cdot l}^n,\quad \pi_{\cdot l}^n:= \sum_i c_{il}^{\cdot n}.\nonumber
\end{gather}

2

\end{document}