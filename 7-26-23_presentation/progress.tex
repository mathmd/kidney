% !TEX program = xelatex
\documentclass{beamer}
\usefonttheme[onlymath]{serif}
\setbeamerfont{footnote}{size=\tiny}

% use multiple languages
\input{../preambles/polyglot.tex}

%% ---- Title page details ---- %%
\title[]{A time-dependent central core model and a schematic model for passive mechanism}
\author[C. Sint]{Chanoknun Sintavanuruk 
% \\ (ชนกนันท์ สินธวานุรักษ์; 馬予棟) \inst{1} 
% \and author2 \inst{2}
}
% \institute[shortinst]{\inst{1} affiliation 
% % \and \inst{2} affiliation for author2
% }
\date{\today}

% preambles for Beamer
\input{../preambles/beamer.tex}

\setbeamerfont{caption}{size=\scriptsize}

\begin{document}
% Title page frame
\begin{frame}
    \titlepage 
\end{frame}
% Remove logo from the next slides
% \logo{}

% Outline frame
\begin{frame}{Outline}
    \tableofcontents
\end{frame}

\section{Time-dependent model}

\subsection{Model equations}

\begin{frame}{Compartments}
    Multiphasic medulla on the domain $(0,L)$ (superficial $\to$ deep).
    \begin{enumerate}
        \item Central core ($k=0$)
        \item Descending tubule ($k=\mathrm{D}$)
        \item Ascending tubule ($k=\mathrm{A}$)
        \item Collecting tubule ($k=\mathrm{C}$)
    \end{enumerate}
\end{frame}

\begin{frame}{Cross-sectional areas}
    Cross-sectional areas $\alpha_k$ ($\text{cm}^2$) satisfy
    \begin{equation}\label{eq:vol_conserv}
        \sum_k \alpha_k = \alpha_*
    \end{equation}
        where $\alpha_*:(0,L)\to \mathbb{R}_+$ is the fixed medullary cross-sectional area.
    \pause
    \begin{align}\label{eq:volume_dynamics}
        \frac{\partial \alpha_k}{\partial t} + \frac{\partial}{\partial x}\left( \alpha_k u_k \right) &= -\gamma_kw_k,\quad k=\mathrm{D},\mathrm{A},\mathrm{C},\\
        \frac{\partial \alpha_0}{\partial t} + \frac{\partial }{\partial x}\left( \alpha_0 u_0 \right) &= \sum_{k=\mathrm{D},\mathrm{A},\mathrm{C}} \gamma_kw_k.
    \end{align}
    $u_k$: axial flow \textit{velocity} (cm/s); $w_k$: transmural flux (cm/s); $\gamma_k$: total tubular circumferences (cm)
\end{frame}

\begin{frame}{Water flow and transport}
    Poiseuille's equation:
    \begin{equation}
        % \frac{\rho_k u_k}{\alpha_k} = -\frac{\partial p_k}{\partial x},\quad k=0,\mathrm{D},\mathrm{A},\mathrm{C},
        \frac{8\pi\eta_k u_k}{\alpha_k} = -\frac{\partial p_k}{\partial x},\quad k=0,\mathrm{D},\mathrm{A},\mathrm{C},
    \end{equation}
    $\eta_k$: viscosity (mmHg$\cdot$s), $p_k$: pressure (mmHg).
    \pause

    Water transport:
    \begin{equation}
        w_k := \zeta_\mathrm{w}^k\left( \psi_k - \psi_0 \right),\quad  k=\mathrm{D},\mathrm{A},\mathrm{C}
    \end{equation}
    % \pause
    % Osmotic pressure:
    \begin{equation}
        \left. \begin{aligned}
            \psi_k&:=p_k - \pi_k,\\
            \pi_k&:= RT\left( 2c_\mathrm{s}^k+c_\mathrm{u}^k\right),
        \end{aligned} \right\}
        \quad k=0,\mathrm{D},\mathrm{A},\mathrm{C}.
    \end{equation}
    $\zeta_\mathrm{w}^k$: water permeability (mmHg$\cdot$s); $c_i^k$: concentration (mmol/$\text{cm}^3$)
\end{frame}

\begin{frame}{Pressure-compliance relationship}
    Given tubular compliance $\nu_k$ ($\text{mmHg}^{-1}$), we assume
    \begin{equation}
        \nu_k(p_k - p_0) = \frac{\alpha_k}{\bar{\alpha}_k} - 1,\quad k=\mathrm{D},\mathrm{A},\mathrm{C}.
    \end{equation}
    Note that $p_0$ is determined by the medullary volume conservation (\ref{eq:vol_conserv}).
\end{frame}

\begin{frame}{Solute dynamics}
    \begin{align}\label{eq:solute_dynamics}
        \frac{\partial}{\partial t}\left( \alpha_k c_i^k \right)&=-\frac{\partial}{\partial x} f_i^k - \gamma_kg_i^k,\quad k=\mathrm{D},\mathrm{A},\mathrm{C},\\
        \frac{\partial}{\partial t}\left( \alpha_0 c_i^0 \right)&=-\frac{\partial}{\partial x} f_i^0 + \sum_{k=\mathrm{D},\mathrm{A},\mathrm{C}} \gamma_k g_i^k,
    \end{align}
    $f_i^k$: Axial solute flow (mmol/s):
    \begin{equation}
        f_i^k := \alpha_k\left( -D_i^k\frac{\partial c_i^k}{\partial x}+u_kc_i^k \right),\quad k=0,\mathrm{D},\mathrm{A},\mathrm{C}.
    \end{equation}
    $D_i^k$: diffusion coefficient ($\text{cm}^2$/s).

    $g_i^k$: transmural solute flux (mmol/$\text{cm}^2\cdot$s),
\end{frame}

\begin{frame}{solute transport}
    $g_i^k$: transmural solute flux (mmol/$\text{cm}^2\cdot$s), (adapted from \cite{Stephenson1987,Stephenson1989}):
    \begin{equation}
        g_i^k := j_i^k+h_i^k,\quad k=\mathrm{D},\mathrm{A},\mathrm{C},
    \end{equation}
    \begin{equation}
        % \left. \begin{aligned}
        %     j_i^k &= \zeta_i^k\left( \mu_i^k - \mu_i^0 \right)\\
        %     \mu_i^k &:= RT\ln c_i^k,\quad 
        % \end{aligned} \right\}\quad k=0,\mathrm{D},\mathrm{A},\mathrm{C}
        j_i^k = \zeta_i^k\left( c_i^k - c_i^0 \right),\quad k=0,\mathrm{D},\mathrm{A},\mathrm{C}.
    \end{equation}
    $\zeta_i^k$: solute permeability (cm/s).

    $h_i^k$: active transport:
    \begin{equation}
        h_\mathrm{s}^\mathrm{A} = \begin{cases}
            \frac{{h}_\mathrm{s,max}^\mathrm{A}}{1+{M}/{c}_\mathrm{s}^\mathrm{A}}\quad &\text{in}\quad (0,\frac{L}{3})\\
            0\quad &\text{in}\quad [\frac{L}{3},L),
        \end{cases}
    \end{equation}
    where $M$ is the Michaelis-Menten constant.
\end{frame}

\begin{frame}{Boundary condition: central core}
    No flux at the bottom:
    \begin{align}
        u_0(t,L) &= 0,\\ 
        f_i^0(t,L)&=0,\quad i=\mathrm{s},\mathrm{u}.
    \end{align}
    Dirichlet boundary at the cortico-medullary junction:
    \begin{align}
        p_0(t,0) &= P_\mathrm{v}(t),\\ 
        c_i^0(t,0) &= c_i^\mathrm{v}(t).
    \end{align}
\end{frame}

\begin{frame}{Boundary condition: descending tubule}
    Input from the PCT:
    \begin{align}
        (\alpha_\mathrm{D} u_\mathrm{D})(t,0) &= F_\mathrm{PCT}(t),\\
        % f_i^\mathrm{D}(t,0) &= (\alpha_\mathrm{D}u_\mathrm{D}c_i^\mathrm{D})(t,0),\quad i=\mathrm{s,u}\\
        c_i^\mathrm{D}(t,0) &= c_{i}^{\mathrm{PCT}}(t),\quad i=\mathrm{s,u},
    \end{align}
    Tip of the loop of Henle:
    \begin{align}
        (\alpha_\mathrm{D}u_\mathrm{D}+\alpha_\mathrm{A}u_\mathrm{A})(t,L) &= 0,\\
        \quad\left( f_i^\mathrm{D}+f_i^\mathrm{A} \right)(t,L) &= 0,\\
        p_\mathrm{D}(t,L)&= p_{\mathrm{A}}(t,L),\\ 
        c_i^\mathrm{D}(t,L) &=c_i^\mathrm{A}(t,L)
    \end{align}
\end{frame}

\begin{frame}{Boundary condition: modification by distal tubules}
    Assumption: salt from the ascending tubule is further reabsorbed so that only a fraction of $q\in (0,1)$ are left at the collecting tubule; formally,
    \begin{align}
        \left( f_{\mathrm{u}}^\mathrm{A}+f_\mathrm{u}^\mathrm{C} \right)(t,0) &= 0,\\
        \left( qf_{\mathrm{s}}^\mathrm{A}+f_\mathrm{s}^\mathrm{C} \right)(t,0) &= 0.
    \end{align}
    Further, we assume that 
    \begin{align}
        % (\alpha_\mathrm{A}u_\mathrm{A}+\alpha_\mathrm{C}u_\mathrm{C})(t,0) &= 0,\\
        % \quad\left( f_i^\mathrm{A}+f_i^\mathrm{C} \right)(t,0) &= 0,\\
        p_\mathrm{A}(t,0)&= p_{\mathrm{C}}(t,0),\\ 
        % c_i^\mathrm{A}(t,0) &=c_i^\mathrm{C}(t,0),
        % \frac{\partial c_i^\mathrm{A}}{\partial x}(t,0)&= 0,\quad i = \mathrm{s},\mathrm{u},
        f_i^\mathrm{A}(t,L) &= (\alpha_\mathrm{A}  u_\mathrm{A}  c_i^\mathrm{A})(t,L),\quad i = \mathrm{s},\mathrm{u},\\ 
        (2c_\mathrm{s}^\mathrm{C}+c_\mathrm{u}^\mathrm{C})(t,0) &=\mathrm{osm}_\mathrm{cortex}(t),
    \end{align}
        where the cortical osmolarity $\mathrm{osm}_\mathrm{cortex}$ is given.
\end{frame}

\begin{frame}{Boundary condition: papillary outflow}
    \begin{align}
        % (\alpha_\mathrm{C} u_\mathrm{C} )(t,L) &= \max\left\{ 0,\frac{p_\mathrm{C} (t,L) - P_{\mathrm{p}} }{R_{\mathrm{p}}}\right\},\\
        p_\mathrm{C} (t,L) &= P_{\mathrm{p}}(t),\\
        f_i^\mathrm{C}(t,L) &= (\alpha_\mathrm{C}  u_\mathrm{C}  c_i^\mathrm{C})(t,L),\quad i=\mathrm{s},\mathrm{u},
    \end{align}
        where $P_\mathrm{p}$ is the papillary pressure.
\end{frame}

% \begin{frame}[Boundary condition: papillary outflow (2/2)]
%     {\color{red} Or we can replace (\ref{eq:no_diff_pap}) by
%     \begin{equation}
%         c_i^\mathrm{C}(t,L) = K*\left( f_i^\mathrm{C}\left( \cdot,L \right) \right)(t)=\int_{\mathbb{R}}K(t-s) f_i^\mathrm{C}(s,L)\,ds,
%     \end{equation}
%     where $K\in L^1(\mathbb{R})$ is non-negative, monotone decreasing in $\mathbb{R}_+$, and $K(t)=0$ for $t<0$.}
% \end{frame}

\subsection{Non-dimensionalization \& numerical simulation}
\begin{frame}{Rescaling}
    Introduce spatial rescaling and advective timescale:
    \begin{equation}
        x = L\hat{x},\quad t = \tau\hat{t},\quad \tau:=\frac{L}{p_*/\rho_*L}=\frac{8\pi\eta_*L^2}{\bar{\alpha}c_*RT}
    \end{equation}
        where the subscript $*$ denotes the typical magnitude of physical quantities; here $p_* = c_*RT$ and $\rho_* = 8\pi\eta_*/\bar{\alpha}$ are those of pressure and hydraulic resistivity with $\bar{\alpha} = \frac{1}{L}\int_0^L\alpha_*(x)\,dx$.

    Unknowns:
    \begin{gather}
        \alpha_k = \bar{\alpha}\hat{\alpha},\quad c_i^k = c_*\hat{c}_i^k,\quad p_k = p_*\hat{p}_k = c_*RT\hat{p}_k.
    \end{gather}    
\end{frame}

\begin{frame}{Dimensionless model}
    \begin{align}
        \frac{\partial \hat{\alpha}_k}{\partial \hat{t}}  + \frac{\partial}{\partial \hat{x}}\left( \hat{\alpha}_k \hat{u}_k \right) &= - \hat{w}_k,\\ \label{eq:nd_1steq}
        \frac{\partial\hat{\alpha}_0}{\partial \hat{t}}+\frac{\partial}{\partial \hat{x}}\left( \hat{\alpha}_0 \hat{u}_0 \right) &=\sum_k \hat{w}_k,\\
        \hat{\nu}_k\left( \hat{p}_k - \hat{p}_0 \right) &= \frac{\hat{\alpha}_k}{\hat{\bar{\alpha}}_k}-1,\\
        \hat{\alpha}_0 + \sum_{k} \hat{\alpha}_k &= \hat{\alpha}_*,\\
        \frac{\partial}{\partial \hat{t}}\left( \hat{\alpha}_k \hat{c}_i^k \right)&=-\frac{\partial}{\partial \hat{x}} \hat{f}_i^k - \hat{g}_i^k,\\
        \frac{\partial}{\partial \hat{t}}\left( \hat{\alpha}_0 \hat{c}_i^0 \right)&=-\frac{\partial}{\partial \hat{x}} \hat{f}_i^0 + \sum_k \hat{g}_i^k,
    \end{align}
\end{frame}

\begin{frame}{Dimensionless flows and transports}
    \begin{align}
        \hat{u}_{k} &:= -\frac{\hat{\alpha}_{k}}{\hat{\rho}_{k}}\frac{\partial \hat{p}_{k}}{\partial \hat{x}},\\
        \hat{f}_i^{k} &:= \hat{\alpha}_{k}\left( -\hat{D}_i^{k} \frac{\partial \hat{c}_i^{k}}{\partial \hat{x}} + \hat{u}_{k}\hat{c}_i^{k} \right),\\
        \hat{w}_k&:= \hat{\zeta}_\mathrm{w}^k\left( \hat{\psi}_k-\hat{\psi}_0 \right),\quad\hat{\psi}_{k} := \hat{p}_{k} - \left(  2\hat{c}_\mathrm{s}^{k}+\hat{c}_\mathrm{u}^{k} \right),\\
        \hat{g}_i^k &:= \hat{j}_i^k+\hat{h}_i^k,\quad \hat{j}_i^k =\hat{\zeta}_i^k(\hat{c}_i^k-\hat{c}_i^0). \label{eq:nd_lasteq}\\
        \hat{h}_\mathrm{s}^\mathrm{A} &= \begin{cases}
            \frac{\hat{h}_\mathrm{s,max}^\mathrm{A}}{1+\hat{M}/\hat{c}_\mathrm{s}^\mathrm{A}}\quad &\text{in}\quad (0,\frac{1}{3})\\
            0\quad &\text{in}\quad [\frac{1}{3},1)
        \end{cases}
    \end{align}
\end{frame}

\begin{frame}{Parameters and boundary conditions}
    Parameters (note that $\gamma_k$ is absorbed into $\hat{\zeta}^k_i,\hat{\zeta^k_{\mathrm{w}}}$):
    \begin{gather*}
        % \mathrm{Pe} = \frac{\bar{\alpha}p_*/\rho_*}{D_*},\quad 
        \hat{\rho}_{k} = \frac{8\pi\eta_{k}}{\rho_*},\quad 
        \hat{\nu}_k = p_*\nu_k,\quad \hat{\bar{\alpha}}_{k} = \frac{\bar{\alpha}_{k}}{\bar{\alpha}},\quad \hat{\alpha}_* = \frac{\alpha_*}{\bar{\alpha}}\\
        \hat{D}_i^{k} = \frac{\tau}{L^2}D_i^{k},\quad 
        % \hat{\zeta}_\mathrm{w}^k = \frac{\gamma_k c_*L^2}{\bar{\alpha}D_*}\zeta_\mathrm{w}^k,\quad\hat{\zeta}_i^k = \frac{\gamma_kRT L^2}{\bar{\alpha}c_* D_*}\zeta_i^k,
        \hat{\zeta}_\mathrm{w}^k = \frac{\gamma_k p_*\tau}{\bar{\alpha}}\zeta_\mathrm{w}^k,\quad\hat{\zeta}_i^k = \frac{\gamma_kp_*\tau}{\bar{\alpha}c_*^2}\zeta_i^k,\\
        \hat{h}_\mathrm{s}^\mathrm{A} = \frac{\gamma_\mathrm{A}\tau}{\bar{\alpha}c_*}h_\mathrm{s}^\mathrm{A},\quad \hat{M} = \frac{M}{c_*}.
    \end{gather*}
    Boundary conditions are the same but with $\hat{\cdot}$ notation where
    \begin{gather*}
        F_\mathrm{PCT} = \frac{\bar{\alpha}L}{\tau}\hat{F}_\mathrm{PCT},\quad
        P_\mathrm{p} = p_*\hat{P}_\mathrm{p},\quad
        P_\mathrm{v} = p_*\hat{P}_\mathrm{v},\\
        c_i^\mathrm{v} = c_*\hat{c}_i^\mathrm{v},\quad \mathrm{osm}_\mathrm{cortex} = c_*\widehat{\mathrm{osm}}_\mathrm{cortex}.
    \end{gather*}

\end{frame}

\begin{frame}{Numerical simulation}
    \begin{itemize}
        \item Approximation: backward difference for time derivatives and central difference for spatial derivatives. Then, use an iterative method.
        \item Most parameters except for $D_i^k$ when $k\neq 0$, $\eta_k$, $P_\mathrm{v}$, $P_\mathrm{p}$, $\nu_k$, $q$ are available in \cite{Stephenson1987,Stephenson1989} which uses transfusion data in rabbit; these 6 will be chosen empirically.
        \item The simulated solution appears to converge to a steady state.
    \end{itemize}
\end{frame}

\begin{frame}{Parameters}
    Physical and geometric parameters (take $\bar{\alpha}_k = \pi r^2$, $\gamma_k = 2\pi r_k$, $\alpha_* = \sum_k \bar{\alpha}_k$):

    \resizebox{\textwidth}{!}{
    \begin{tabular}{||c c c|c|c|c c c c||} 
        \hline
        $D_\mathrm{s}^0$ & $D_\mathrm{u}^0$ & $D_i^k$, $k\neq 0$ & $\eta_k$ & $\nu_k$ & $r_0$ & $r_\mathrm{D}$ & $r_\mathrm{A}$ & $r_\mathrm{C}$\\
        \hline
        \multicolumn{3}{||c|}{{\tiny($\times 10^{-4}\,\text{cm}^2$/s)}} & {\tiny (cP $\approx 7.5\times 10^{-6}$ mmHg$\cdot$s)} & {\tiny ($\text{mmHg}^{-1}$)} & \multicolumn{4}{|c||}{{\tiny ($\times 10^{-3}$ cm)}}\\
        \hline\hline
        2.5 & 2 & 0.15 & 0.6915 & 0.0517 & 2.5 & 0.8 & 1 & 1.2 \\ %[1ex] 
        \hline
       \end{tabular}       
    }
    
    Transport parameters:

    \resizebox{\textwidth}{!}{
    \begin{tabular}{||c c c|c c c|c c c|c c||} 
        \hline
        $\zeta_\mathrm{w}^\mathrm{D}$ & $\zeta_\mathrm{w}^\mathrm{A}$ & $\zeta_\mathrm{w}^{\mathrm{C}}$ &
         $\zeta_\mathrm{s}^\mathrm{D}$ & $\zeta_\mathrm{s}^\mathrm{A}$ & $\zeta_\mathrm{s}^\mathrm{C}$ & 
         $\zeta_\mathrm{u}^\mathrm{D}$ & $\zeta_\mathrm{u}^\mathrm{A}$ & $\zeta_\mathrm{u}^\mathrm{C}$ & $h_\mathrm{s}^{\mathrm{A}}$ & $M$ \\
        \hline
        \multicolumn{3}{||c|}{{\tiny ($\times 10^{-8}$ cm/mmHg$\cdot$s)}} & \multicolumn{6}{|c|}{{\tiny ($\times 10^{-5}$ cm/s)}} & {\tiny ($\frac{10^{-6}\text{mmol}}{\text{cm}^2\cdot\text{s}}$)} & {\tiny ($\frac{\text{mmol}}{\text{cm}^{3}}$)} \\ %[1ex] 
        \hline\hline
        22.5 & 0 & 3.95 & 1.61 & 6.27 & 0.04 & 1.5 & 0.86 & 0 & 14.2 & 0.15 \\ %[1ex] 
        33.8 & 0 & 3.95 & 1.61 & 26.0 & 0.04 & 1.5 & 6.70 & 0 & 0 &  \\ %[1ex] 
        26.4 & 0 & 3.95 & 1.61 & 26.0 & 0.04 & 1.5 & 6.70 & 1.5 & 0 &  \\ %[1ex] 
        \hline
    \end{tabular}   
    }

\end{frame}

\begin{frame}{Boundary data}
    \begin{tabular}{||c|c c|c c c||} 
        \hline
        $F_\mathrm{PCT}$ & $P_\mathrm{p}$ & $P_\mathrm{v}$ & $c_\mathrm{s}^\mathrm{v}$ & $c_\mathrm{u}^{\mathrm{v}} $ & $\mathrm{osm}_\mathrm{cortex}$ \\
        \hline
        {\tiny ($\times 10^{-7}\text{cm}^{3}$/s)} & \multicolumn{2}{|c|}{{\tiny (mmHg)}} & \multicolumn{3}{|c||}{{\tiny (mmol/L)}}\\
        \hline\hline
        1.67 & 6.4 & 0 & 145 & 5 & 295 \\ %[1ex] 
        \hline
       \end{tabular}       
\end{frame}
    
\begin{frame}{Result}
    
    \begin{figure}
        \centering
        \includegraphics[width=\textwidth]{../results/6-6-2023/osm.png}
    \end{figure}
\end{frame}

\section{Schematic model for passive mechanism}

\begin{frame}{Idea}
    \begin{itemize}
        \item We want to have a clean picture of what makes passive mechanism working.
        \item Derive a schematic model of \textit{inner medulla} based on the common explanation of passive mechanism:
    \end{itemize}
    \begin{alertblock}{Concept of passive mechanism}
        The net water and NaCl reabsorption preceding the inner medulla collecting tubules concentrate the urea enough that it diffuses out in the inner medulla.
        This in turn increases osmolality in the interstitium that drives the water reabsorption from the thin descending limbs, concentrating NaCl in the process.
        NaCl is then passively reabsorbed at the ascending tubule.
    \end{alertblock}
    
\end{frame}

\begin{frame}{Model derivation}
    Consider a steady state of the previous dimensionless model but rescale $\hat{x}\in (0,1)$ to be the inner medulla instead (from now on, we omit the $\hat{\cdot}$ notation). 
    Writing $q_k = \alpha_ku_k$, $s_k = c_\mathrm{s}^k$, $u_k = c_\mathrm{u}^k$, and assuming that $1/\nu_k$ and $D_i^k$ are small.
    (continued in the document)
\end{frame}


\begin{frame}[allowframebreaks]{References}
    \bibliographystyle{plainnat}
    \tiny\bibliography{../bibliography}
\end{frame}

\end{document}